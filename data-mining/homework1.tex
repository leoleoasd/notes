\documentclass{ctexart}
\usepackage[T1]{fontenc}
\usepackage[a4paper,top=1.5cm,bottom=1.5cm,left=2cm,right=2cm,marginparwidth=1.75cm]{geometry}
\usepackage{mathtools}
\usepackage{tikz}
\usepackage{booktabs}
\usepackage{caption}
\usepackage{outlines}
\usepackage{graphicx}
\usepackage{float}
\usepackage{amsthm}
\usepackage{tabularray}
\usepackage{minted}
\usepackage[colorlinks=false, allcolors=blue]{hyperref}
\usepackage{cleveref}
\renewcommand{\tableautorefname}{表}
\DeclarePairedDelimiter{\set}{\{}{\}}
\DeclarePairedDelimiter{\paren}{(}{)}
\graphicspath{ {./images/} }

\xeCJKsetup{CJKmath=true}

\newcounter{fullrefcounter}
\newcommand*{\fullref}[1]{%
\addtocounter{fullrefcounter}{1}%
\label{--ref-\thefullrefcounter}%
\ifthenelse{\equal{\getpagerefnumber{--ref-\thefullrefcounter}}{\getpagerefnumber{#1}}}
  {
    \hyperref[{#1}]{\Cref*{#1} \nameref*{#1}}
  }
  {% false case
    \hyperref[{#1}]{第 \pageref*{#1} 页 \Cref*{#1} \nameref*{#1}}
  }
}

\title{数据挖掘第一次作业}
\author{卢雨轩 19071125}
% \date{\today}
\ctexset{
    section = {
        titleformat = \raggedright,
        name = {,},
        number = \chinese{section}、
    },
    paragraph = {
        runin = false
    },
    today = small,
    figurename = 图,
    contentsname = 目录,
    tablename = 表,
}

\begin{document}

\maketitle

\begin{enumerate}
  \item[4-1] 假定两个水平类似的班级(一班和二班)上同一门课,
  两位任课老师的评分标准有差异,两个班成绩的均值和标准差都不一样。
  一班分数的均值和标准差分别为78.53和9.43,二班的均值和标准差分别为70.19和7.00。

  那么得到90分的一班的张颖和得到82分的二班的刘疏哪个成绩更好呢?说明理由

    答:张颖同学的标准分数$Z_{张颖} = \frac{x_{张颖} - \mu}{\delta} = \frac{90 - 78.53}{9.43} \approx 1.21$

    刘疏的标准分数$Z_{刘疏} = \frac{x_{刘疏} - \mu}{\delta} = \frac{82 - 70.19}{7} \approx 1.68 > Z_{张颖}$。

    因此刘疏同学成绩更好。
  
    \item[4-2] 《大西洋月刊》的一篇文章中谈到了IQ值(智商)与出生率。IQ值呈钟形分布,其平均数为100,标准差为15。
    \begin{enumerate}
      \item IQ值在85\textasciitilde 115之间的人口所占百分比为多少?
      
      IQ值在85\textasciitilde 115之间,也就是在$\pm 15$范围内的人口占$68\%$。
      \item IQ值在70\textasciitilde 130之间的人口所占百分比为多少?
      
      IQ值在 70 \textasciitilde 130 之间,也就是在$\pm 30$范围内的人口占$95\%$。

      \item 凡是IQ值超过145的人都被视为天才,经验法则是否支持这一论断?试解释原因。
      
      经验法则支持本结论。IQ值在 55 \textasciitilde 145 之间,也就是在$\pm 45$范围外的人口占$1\%$,因此IQ值超过145的人可认为超越了$99.5\%$的人,可被视为天才。
    \end{enumerate}
\end{enumerate}

\end{document}
