\documentclass{ctexart}
\usepackage[T1]{fontenc}
\usepackage[a4paper,top=1.5cm,bottom=1.5cm,left=2cm,right=2cm,marginparwidth=1.75cm]{geometry}
\usepackage{mathtools}
\usepackage{tikz}
\usepackage{booktabs}
\usepackage{caption}
\usepackage{outlines}
\usepackage{graphicx}
\usepackage{amsthm}
\usepackage[colorlinks=false, allcolors=blue]{hyperref}
\renewcommand{\tableautorefname}{表}
\DeclarePairedDelimiter{\set}{\{}{\}}
\DeclarePairedDelimiter{\paren}{(}{)}
\graphicspath{ {./images/} }

\title{形式语言第三次作业}
\author{卢雨轩 19071125}
% \date{\today}
\ctexset{
    section = {
        titleformat = \raggedright,
        name = {,},
        number = \chinese{section}、
    },
    paragraph = {
        runin = false
    },
    today = small,
    figurename = 图,
    contentsname = 目录,
    tablename = 表,
}

\begin{document}

\maketitle

\begin{outline}[enumerate]
    \1[4.] 下列各式成立吗?请证明你的结论。
        \2[(3)] $rs = sr$
        
        不成立。如:令$r = a,\;s = b$。则$rs$识别的语言为$\set{ab}$,$sr$
        识别的语言为$\set{ba}$。
        \2[(6)] $(r+s)^* = r^* + s^*$

        不成立。如:令$r = a,\;s = b$。则$rs$识别的语言为$a,b$组成的串,$sr$
        识别的语言为$a$组成的串或$b$组成的串。
\end{outline}

\end{document}
