\documentclass{ctexart}
\usepackage[T1]{fontenc}
\usepackage[a4paper,top=1.5cm,bottom=1.5cm,left=2cm,right=2cm,marginparwidth=1.75cm]{geometry}
\usepackage{mathtools}
\usepackage{tikz}
\usepackage{booktabs}
\usepackage{caption}
\usepackage{outlines}
\usepackage{graphicx}
\usepackage{amsthm}
\usepackage[colorlinks=false, allcolors=blue]{hyperref}
\renewcommand{\tableautorefname}{表}
\DeclarePairedDelimiter{\set}{\{}{\}}
\DeclarePairedDelimiter{\paren}{(}{)}
\graphicspath{ {./images/} }

\title{形式语言第四次作业}
\author{卢雨轩 19071125}
% \date{\today}
\ctexset{
    section = {
        titleformat = \raggedright,
        name = {,},
        number = \chinese{section}、
    },
    paragraph = {
        runin = false
    },
    today = small,
    figurename = 图,
    contentsname = 目录,
    tablename = 表,
}

\begin{document}

\maketitle

\begin{outline}[enumerate]
    \1[2.] 下列语言都是字母表$\Sigma = \set{0, 1}$上的语言。他们哪些是RL,哪些不是RL?
        \2[(1)] $\set{0^2n | n \ge 1}$

            是RL。可构造RE=$00(00)^*$
        \2[(3)] $\set{0^n1^m0^n | n, m \ge 1}$

            不是RL。

            设$L$是RL。$N$是泵引理所说正整数。

            取$z = uvw = 0^N10^N$。令$v = 0^x$。所以$u = 0^{N-x}$,$w = 10^N$。

            当k = 0时,有$uv^kw = 0^{N-x}10^N \ne L$,与泵引理矛盾。

            所以L不是RL。

    \1[11.] 
        \2[(3)] $L(M) = L(M_1) - L(M_2)$
            $M = (Q_1 \times Q_2, \Sigma, \delta, [q_{01}, q_{02}], F_1 \times (Q_2 - F_2))$

            $\delta([q, p], a) = [\delta_1(q, a), \delta_2(p, a)]$
        \2[(4)] $L(M) = L(M_1) \cup L(M_2)$
        $M = (Q_1 \times Q_2, \Sigma, \delta, [q_{01}, q_{02}], Q_1 \times F_2 \cup Q_2 \times F_1)$

        $\delta([q, p], a) = [\delta_1(q, a), \delta_2(p, a)]$
\end{outline}

\end{document}
