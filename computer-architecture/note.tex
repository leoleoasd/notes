% !TEX program = xelatex
% !TEX options = -shell-escape -synctex=1 -interaction=nonstopmode -file-line-error "%DOC%"
\documentclass{ctexart}
\usepackage[a4paper,top=1.5cm,bottom=1.5cm,left=2cm,right=2cm,marginparwidth=1.75cm]{geometry}
\usepackage{amsmath}
\usepackage{booktabs}
\usepackage{caption}
\usepackage[colorlinks=false, allcolors=blue]{hyperref}
\usepackage{outlines}
\renewcommand{\tableautorefname}{表}

\title{计算机系统结构}
\author{卢雨轩 19071125}
% \date{\today}
\ctexset{
    section = {
        titleformat = \raggedright,
        name = {,},
        number = 第\chinese{section}章
    },
    paragraph = {
        runin = false
    },
    today = small,
    figurename = 图,
    contentsname = 目录,
    tablename = 表,
}

\begin{document}

\maketitle
\tableofcontents

\section{指令系统}
\subsection{数据类型和表示}

\section{流水线技术}
\subsection{流水线处理概述}
\subsection{流水工作方式}

\begin{outline}
    \1 流水线处理概念和特点
        \2 流水线技术:将一个重复的时序过程分成若干个子过程,每个子过程都可有效的在其专用功能段上和其它子过程同时执行的一种技术。 
        \2 流水线特点:
            \3 流水一定重叠,比重叠更苛刻。
            \3 一条流水线通常由多个流水段组成。
            \3 每段有专用功能部件,各部件顺序连接,不断流。
            \3 流水线有建立时间、满载时间、排空时间,
            \3 各段时间尽量短、一致,不一致时最慢子过程为瓶颈。
            \3 给出的最大吞吐率等指标,为满负载最佳指标。
    \1 流水线的分级、分类
        \2 流水线的分级
            \3 操作部件级(arithmetic pipelining):将复杂的算逻运算组成流水工作方式; 
            \3 指令级(instruction pipelining) :把一条指令解释过程分成多个子过程 ;
            \3 处理机级或宏流水线级(macro pipelining) :由两个以上处理机串行地对同一数据流进行处理,每个处理机完成某一专门任务,各个处理机所得到的结果需存放在与下一个处理机所共享的存储器中。 
        \2 流水线的分类
            \3 按照功能分类
                \4 单功能流水线
                \4 多功能流水线
                \4 主要用单功能
            \3 按工作方式
                \4 静态
                \4 动态
            \3 按链接方式
                \4 线性流水线
                \4 非线性流水线
    \1 流水线性能分析计算
        \2 技术指标
            \3 吞吐率(TP,Throughput Rate):单位时间内能处理的指令条数或能输出的数据量。
                \4 最大吞吐律
                \4 『瓶颈』子过程
            \3 效率(Efficiency):流水线的设备利用率;
            \3 加速比
        \2 实例分析:性能分析(实测法,  分析法,  时空图法)
    \1 流水的控制和设计
        \2 时序和缓冲
            \3 △t尽量小、尽量一致
            \3 △t选取:各级微操作所需逻辑门的延迟、加走线延迟、再加3%冗余量。
            \3 缓冲深度影响(缓冲器个数)。
            \3 非线性流水的合理控制。
        \2 相关处理
            \3 资源相关
            \3 数据相关:
                \4 read after write
                \4 write after write
                \4 write after read
                \4 时间推后法
                \4 旁路技术或相关专用通路技术(定向技术)
                \4 定向技术:将一个计算结果直接传送到所有需要它的功能单元的输入端。
            \3 控制相关

\end{outline}

\end{document}