% !TEX program = xelatex
% !TEX options = -shell-escape -synctex=1 -interaction=nonstopmode -file-line-error "%DOC%"
\documentclass{ctexart}
\usepackage[a4paper,top=1.5cm,bottom=1.5cm,left=2cm,right=2cm,marginparwidth=1.75cm]{geometry}
\usepackage{amsmath}
\usepackage{booktabs}
\usepackage{caption}
\usepackage[colorlinks=false, allcolors=blue]{hyperref}
\usepackage{outlines}
\renewcommand{\tableautorefname}{表}

\title{计算机系统结构}
\author{卢雨轩 19071125}
% \date{\today}
\ctexset{
    section = {
        titleformat = \raggedright,
        name = {,},
        number = 第\chinese{section}章
    },
    paragraph = {
        runin = false
    },
    today = small,
    figurename = 图,
    contentsname = 目录,
    tablename = 表,
}

\begin{document}

\maketitle
\tableofcontents

\section{指令系统}
\subsection{计算机系统设计}
\begin{outline}
    \1 计算机系统的设计原则
        \2 加速那些使用频率高的部件
        \2 Amadahl定律
            \3 系统中某一个部件改进后性能提高,用加速比衡量
            \3 加速比等于(改进前运行时间/改进后运行时间)
            \3 Sp:加速比;
            \3 Te:采用改进措施前执行某任务系统所用的时间;
            \3 T0:采用改进措施后所需的时间;
            \3 fe:可改进部分在原系统计算时间中所占的比例,总是小于1;
            \3 re:性能提高的倍数(T部件改进前/ T部件改进后),总是大于1。
        \2 程序的访问局部性原理
            \3 时间:近期访问的代码,之后可能再次访问
            \3 空间:地址相邻的代码可能连续访问
    \1 计算机系统结构设计方法:
        \2 从下而上:基本不用
        \2 从上而下:适合专用机
        \3 从中间:最好
\end{outline}
\subsection{计算机的性能评价}
\begin{outline}
    \1 主频,子长,主存容量
    \1 运算速度
        \2 CPI:Cycle per Instruction
            \3 总周期数/指令数
        \2 MIPS:每秒执行多少百万条指令
            \3 条数/(执行时间$\times 10^-6$)
            \3 时钟频率/(CPI $\times 10^-6$)
        \2 MFLOPS:每秒执行多少百万条浮点运算
            \3 浮点运算次数/(执行时间$\times 10^-6$)
    \1 兼容性:
        \2 软件兼容:
        \3 上下兼容:兼容更高或更低档次的计算机
        \3 前后兼容:兼容先、后设计的计算机
\end{outline}
\subsection{要点}
\begin{outline}
    \heiti
    \1 计算机层次结构
    \1 计算机系统结构
    \1 弗林分类法
    \1 CPU 性能
\end{outline}

\section{指令系统}
\begin{outline}
    \1 指令系统,指令集
        \2 指令格式
        \2 数据类型
            \3 自定义的数据表示
            \3 带标识符的数据表示
                \4 数据值和每个标识符相连
                \4 数据字长增加
            \3 数据描述符:
                \4 描述多位和复杂数据
                \4 和数据字分开储存
                \4 增加一层寻址
        \2 操作功能
        \2 操作数访问方式
    \1 指令编码方法
        \2 正交法
        \2 整体法
        \2 混和法
\end{outline}
\subsection{要点}
\begin{outline}
    \heiti
    \1   数据类型和数据表示
    \1   自定义数据表示定义、分类及优点
    \1   哈夫曼概念及在计算机中应用,操作码编码法 
    \1   RISC的主要技术 
\end{outline}
\section{储存系统}
\subsection{储存系统概念}
\begin{outline}
    \1 包含性:在大容量储存器中一定能找到上层储存器内容的副本
    \1 一致性:副本修改,保证同一信息的一致性
\end{outline}
\subsection{CACHE}
\begin{outline}
    \1 地址映像规则
        \2 全相联方式
            \3 主存和缓存分成大小相同的数据块,每一块可以装入任意一块的空间中
            \3 地址:块号+块内地址
            \3 目录表:内存块号+缓存块号+有效位
        \2 直接相联
            \3 主存和缓存分成大小相同的数据块
            \3 主存容量应该是缓存容量的整数被,被分成很多区,每区和缓存总块数相等
            \3 某区的一块进入缓存时只能存入缓存中块号相同的位置
            \3 目录表:区号+有效位(每个缓存块对应一个目录)
        \2 组相联
            \3 主存和缓存分成大小相同的数据块
            \3 主存和缓存分成大小相同的组
            \3 组间全相联,组内直接相联
            \3 地址:区号+区内组号+组内块号+块内地址
            \3 目录表:内存区号+内存块号+缓存块号+有效位
\end{outline}
\subsection{要点}
\begin{outline}
    \heiti
    \1 存储系统的概念
    \1 高速缓冲存储器
    \1 地址映像与变换
    \1 LRU替换算法
    \1 存储保护
\end{outline}

\section{流水线技术}
\subsection{流水线处理概述}
\subsection{流水工作方式}

\begin{outline}
    \1 流水线处理概念和特点
        \2 流水线技术:将一个重复的时序过程分成若干个子过程,每个子过程都可有效的在其专用功能段上和其它子过程同时执行的一种技术。 
        \2 流水线特点:
            \3 流水一定重叠,比重叠更苛刻。
            \3 一条流水线通常由多个流水段组成。
            \3 每段有专用功能部件,各部件顺序连接,不断流。
            \3 流水线有建立时间、满载时间、排空时间,
            \3 各段时间尽量短、一致,不一致时最慢子过程为瓶颈。
            \3 给出的最大吞吐率等指标,为满负载最佳指标。
    \1 流水线的分级、分类
        \2 流水线的分级
            \3 操作部件级(arithmetic pipelining):将复杂的算逻运算组成流水工作方式; 
            \3 指令级(instruction pipelining) :把一条指令解释过程分成多个子过程 ;
            \3 处理机级或宏流水线级(macro pipelining) :由两个以上处理机串行地对同一数据流进行处理,每个处理机完成某一专门任务,各个处理机所得到的结果需存放在与下一个处理机所共享的存储器中。 
        \2 流水线的分类
            \3 按照功能分类
                \4 单功能流水线
                \4 多功能流水线
                \4 主要用单功能
            \3 按工作方式
                \4 静态
                \4 动态
            \3 按链接方式
                \4 线性流水线
                \4 非线性流水线
    \1 流水线性能分析计算
        \2 技术指标
            \3 吞吐率(TP,Throughput Rate):单位时间内能处理的指令条数或能输出的数据量。
                \4 最大吞吐律
                \4 『瓶颈』子过程
            \3 效率(Efficiency):流水线的设备利用率;
            \3 加速比
        \2 实例分析:性能分析(实测法,  分析法,  时空图法)
    \1 流水的控制和设计
        \2 时序和缓冲
            \3 △t尽量小、尽量一致
            \3 △t选取:各级微操作所需逻辑门的延迟、加走线延迟、再加3%冗余量。
            \3 缓冲深度影响(缓冲器个数)。
            \3 非线性流水的合理控制。
        \2 相关处理
            \3 资源相关
            \3 数据相关:
                \4 read after write
                \4 write after write
                \4 write after read
                \4 时间推后法
                \4 旁路技术或相关专用通路技术(定向技术)
                \4 定向技术:将一个计算结果直接传送到所有需要它的功能单元的输入端。
            \3 控制相关

\end{outline}

\end{document}