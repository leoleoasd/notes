\documentclass{ctexart}

\usepackage{tabularray}
% \usepackage{enumerate}
\usepackage{enumitem}
\UseTblrLibrary{booktabs}
\setlist[itemize]{leftmargin=4\ccwd}
% \setlist{leftmargin=4\ccwd}
% \setlength{\leftmargin}{120em}

\begin{document} 
\title{\vspace{-3.8cm}EduOJ 需求分析}
\author{卢雨轩~19071125\\ 孙天天19071110}
\date{2022年3月14日}

\maketitle

\section{软件的用户}
EduOJ作为一款面向教学的在线评测系统,其用户主要分为以下三类:
\begin{itemize}
    \item \texttt{全局管理员}:管理整个系统。
    \item \texttt{教师}:管理某个班级的学生、作业。
    \item \texttt{学生}:参与某个班级的学习。
\end{itemize}

\section{各种用户具有的功能}
本系统中的各种功能对各类用户拥有的情况如下:
\begin{center}
    \begin{tblr}{c c c c}
        \toprule
        功能名称 & 全局管理员 & 教师 & 学生\\
        \midrule
        查看他人资料 & 有 & 有 & 有 \\
        修改个人资料 & 有 & 有 & 有 \\
        管理用户 & 有 & 有 & \\
        \midrule
        查看题目 & 全部 & 全部 & 班级内 \\
        编辑题目 & 有 & 自己创建的 & \\
        查看提交 & 全部 & 班级内 & 自己的 \\
        创建提交 & 全部 & 班级内 & 班级内 \\
        \midrule
        查看班级成员 & 全部 & 教授的 & 参与的 \\
        管理班级成员 & 全部 & 教授的 & \\
        查看题目组 & 全部 & 教授的 & 参与的 \\
        管理题目组 & 全部 & 教授班级的 & \\
        \midrule
        上传图片 & 有 & 有 & 有 \\
        查阅日志 & 有 & & \\
        添加脚本 & 有 & & \\
        \bottomrule
    \end{tblr}
\end{center}

\section{每个功能的流程说明}

查看他人资料
\begin{itemize}
    \item 请求获取他人资料
    \item 从数据库中查询他人资料
\end{itemize}

修改个人资料
\begin{itemize}
    \item 接收修改后的个人资料
    \item 向数据库更新个人资料
\end{itemize}

管理用户
\begin{itemize}
    \item 确认申请者具有权限
    \item 接收涉及的用户和修改后的资料
    \item 向数据库更新用户资料
\end{itemize}

查看题目
\begin{itemize}
    \item 确认申请者身份、所在班级
    \item 从数据库提取题目资料
\end{itemize}

编辑题目
\begin{itemize}
    \item 确认申请者身份、教授班级
    \item 向数据库更新题目资料
\end{itemize}

查看提交
\begin{itemize}
    \item 确认申请者身份、所在班级、教授班级
    \item 从数据库获取全部或个人的提交资料
\end{itemize}

创建提交
\begin{itemize}
    \item 确认申请者身份、所在班级
    \item 接收代码,使用评测机评测
    \item 整理评测结果,向数据库写入提交资料
\end{itemize}

查看班级成员
\begin{itemize}
    \item 确认申请者身份、所在班级
    \item 从数据库获取指定班级成员
\end{itemize}

管理班级成员
\begin{itemize}
    \item 确认申请者身份、教授班级
    \item 向数据库更新班级成员
\end{itemize}

查看题目组
\begin{itemize}
    \item 确认申请者身份、所在班级
    \item 从数据库获取指定题目组
\end{itemize}

管理题目组
\begin{itemize}
    \item 确认申请者身份、教授班级
    \item 向数据库更新指定题目组
\end{itemize}

上传图片
\begin{itemize}
    \item 接收图片,向数据库添加图片记录
    \item 向文件服务器添加图片文件
\end{itemize}

查阅日志
\begin{itemize}
    \item 接收检索要求
    \item 从数据库提取指定日志
\end{itemize}

添加脚本
\begin{itemize}
    \item 接收脚本,向数据库添加脚本记录
    \item 向文件服务器添加脚本文件
\end{itemize}

\end{document}
