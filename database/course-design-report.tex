\documentclass{ctexrep}
\usepackage[T1]{fontenc}
\usepackage[a4paper,top=1.5cm,bottom=1.5cm,left=2cm,right=2cm,marginparwidth=1.75cm]{geometry}
\usepackage{mathtools}
\usepackage{tikz}
\usepackage{booktabs}
\usepackage{caption}
\usepackage{outlines}
\usepackage{graphicx}
\usepackage{amsthm}
\usepackage{tabu}
\usepackage{minted}
\usepackage[final]{pdfpages}
\usepackage[colorlinks=false, allcolors=blue]{hyperref}
\usepackage{cleveref}
\usepackage{psql}
% \usepackage{markdown}
\usepackage{underscore}
\usepackage{longtable}
\renewcommand{\tableautorefname}{表}
\DeclarePairedDelimiter{\set}{\{}{\}}
\DeclarePairedDelimiter{\paren}{(}{)}
\graphicspath{ {./images/} }
\newcommand*{\fullref}[1]{
\ifthenelse{\equal{\thepage}{\getpagerefnumber{#1}}}
  {
    \hyperref[{#1}]{\Cref*{#1} \nameref*{#1}}
  }
  {% false case
    \hyperref[{#1}]{第 \pageref*{#1} 页 \Cref*{#1} \nameref*{#1}}
  }
}
\definecolor{bg}{rgb}{0.95,0.95,0.95}

\newmintinline[sql]{postgresql}{}

% \title{标题}
% \author{卢雨轩 19071125}
% \date{\today}
\ctexset{
    chapter = {
        titleformat = \raggedright,
        name = {实验,:},
        number = \chinese{chapter}
    },
    section = {
        titleformat = \raggedright,
        name = {,},
        number = \chinese{section}、
    },
    paragraph = {
        runin = false
    },
    today = small,
    figurename = 图,
    contentsname = 目录,
    tablename = 表,
}

\begin{document}

\includepdf{report-cover.pdf}

\tableofcontents
\clearpage
\chapter*{相关说明}
\section*{实验环境:}
PostgreSQL版本:\texttt{psql (PostgreSQL) 12.9 (Ubuntu 12.9-0ubuntu0.20.04.1)}

openGauss版本:\texttt{gsql (openGauss 2.0.0 build 78689da9)}

\section*{数据库设计}

本实验涉及的数据库设计源自\href{https://github.com/EduOJ/backend}{EduOJ}后端数据库的用户、权限、班级管理部分。\href{https://github.com/EduOJ/backend}{EduOJ}的数据库设计由卢雨轩(本实验报告的作者)和孙天天共同完成。复用以往项目作为实验设计的行为已经经过指导老师授权。

\chapter*{数据库设计}

\section{数据需求描述}

\subsection{管理员:}

\begin{itemize}
\item
  用户增删改查
\item
  班级增删改查
\item
  作业增删改查
\item
  成绩查询、修改
\end{itemize}

\hypertarget{ux7528ux6237ux76f8ux5173}{%
\subsubsection{用户相关}\label{ux7528ux6237ux76f8ux5173}}

\begin{itemize}
\item
  姓名,学号,密码,邮箱
\item
  权限相关:

  \begin{itemize}
  \item
    『全局权限』(如:某个用户有权限创建用户、修改用户、创建班级、创建题目)
  \item
    『针对权限』(如:某个用户有针对id为5的班级的添加学生权限)
  \item
    希望能够『批量管理』:把权限授予给『角色』,让『用户』拥有『角色』。
  \end{itemize}
\item
  统计信息,加快查询速度
\end{itemize}

\subsubsection{班级相关}

\begin{itemize}
\item
  课程名称,课堂名称,管理老师,课程描述
\item
  用户、作业

  \begin{itemize}
  \item
    和用户是多对多关系
  \item
    和作业是多对多关系
  \end{itemize}
\end{itemize}

\hypertarget{ux4f5cux4e1aux76f8ux5173}{%
\subsubsection{作业相关}\label{ux4f5cux4e1aux76f8ux5173}}

\begin{itemize}
\item
  标题,起止日期
\item
  多道题目
\item
  统计成绩
\end{itemize}

\hypertarget{ux6210ux7ee9ux67e5ux8be2ux4feeux6539}{%
\subsubsection{成绩查询、修改}\label{ux6210ux7ee9ux67e5ux8be2ux4feeux6539}}

\begin{itemize}
\item
  用户、班级、作业、成绩
\item
  根据用户做题记录生成,用于加快计算,\textbf{有数据冗余}。
\end{itemize}

\section{数据库设计}


\subsection{ER图}

\includegraphics[width=\linewidth]{./course-design-er.pdf}
\captionof{figure}{EduOJ用户、班级、作业、权限管理部分ER图}


\subsection{关系模式}

\begin{itemize}
\item
  user(\underline{id}, nickname, username, email, password, created_at, updated_at)
\item
  roles(\underline{id}, name, target)
\item
  user_has_roles(\underline{id}, user_id, )
\item
  permissions(\underline{id},role_id, name)
\item
  tokens(\underline{id},user_id,token, created_at,remember_me)
\item
  webauthn_credentials(\underline{id}, user_id, content, created_at)
\item
  classes(\underline{id}, name, course_name, description, invite_code)
\item
  user_in_classes(\underline{id}, user_id, class_id)
\item
  user_managing_classes(\underline{id}, user_id, class_id)
\item
  grades(\underline{id}, total, detail, user_id, class_id,
  problem_set_id)
\item
  problem_sets(\underline{id}, name, description,
  start_time,end_time)
\item
  problem_in_problem_sets(\underline{id}, problem_id,
  problem_set_id)
\item
  problems(\underline{id}, name, description, attach_file_name,
  public, memory_limit, time_limit, compare_script_name, build_arg)
\end{itemize}


\subsection{范式判断}


\subsubsection{1NF}

所有关系模式中,属性均是原子的,符合范式。

\subsubsection{2NF}

除了grades的所有关系模式中均依赖主键\underline{id},符合范式。

grades中,class_id依赖problem_set_id、detail和total依赖评测结果(未给出),但是为了加速查询,保留数据冗余。


\subsubsection{3NF}

表中除了主键之外所有属性均不互相依赖,符合范式。

\subsubsection{BCNF}

所有关系模式均只有一个主属性,不存在其他键码,同时非主属性也依赖与键码,所以符合范式。
\subsubsection{4NF}

所有关系模式均不存在平凡多值依赖,故符合4NF。

\section{数据表设计}

\subsection{users}

\begin{longtable}[]{@{}cccc@{}}
\toprule
字段名称 & 类型 & 索引 & 外键\tabularnewline
\midrule
id & bigint & primary &\tabularnewline
username & varchar(30) & index &\tabularnewline
nickname & varchar(30) & index &\tabularnewline
email & varchar(320) & index &\tabularnewline
password & varchar(60) & &\tabularnewline
created_at & timestamp & &\tabularnewline
updated_at & timestamp & &\tabularnewline
\bottomrule
\end{longtable}

\subsubsection{roles}

\begin{longtable}[]{@{}cccc@{}}
\toprule
字段名称 & 类型 & 索引 & 外键\tabularnewline
\midrule
id & bigint & primary &\tabularnewline
name & varchar(255) & &\tabularnewline
target & varchar(255) & index &\tabularnewline
\bottomrule
\end{longtable}

\subsubsection{user_has_roles}



\begin{longtable}[]{@{}cccc@{}}
\toprule
字段名称 & 类型 & 索引 & 外键\tabularnewline
\midrule
id & bigint & primary &\tabularnewline
user_id & bigint & index & users(id)\tabularnewline
role_id & bigint & index & roles(id)\tabularnewline
target_id & bigint & index &\tabularnewline
\bottomrule
\end{longtable}



\hypertarget{permissions}{%
\subsection{permissions}\label{permissions}}



\begin{longtable}[]{@{}cccc@{}}
\toprule
字段名称 & 类型 & 索引 & 外键\tabularnewline
\midrule
id & bigint & primary &\tabularnewline
role_id & bigint & index & roles(id)\tabularnewline
name & varchar(255) & index &\tabularnewline
\bottomrule
\end{longtable}


\subsection{tokens}



\begin{longtable}[]{@{}cccc@{}}
\toprule
字段名称 & 类型 & 索引 & 外键\tabularnewline
\midrule
id & bigint & primary &\tabularnewline
user_id & bigint & index & users(id)\tabularnewline
token & varchar(32) & index &\tabularnewline
created_at & timestamp & &\tabularnewline
remember_me & boolean & &\tabularnewline
\bottomrule
\end{longtable}


\subsection{webauthn_credentials}

\begin{longtable}[]{@{}cccc@{}}
\toprule
字段名称 & 类型 & 索引 & 外键\tabularnewline
\midrule
id & bigint & primary &\tabularnewline
user_id & bigint & index & users(id)\tabularnewline
content & varchar(32) & &\tabularnewline
created_at & timestamp & &\tabularnewline
\bottomrule
\end{longtable}

\subsection{classes}


\begin{longtable}[]{@{}cccc@{}}
\toprule
字段名称 & 类型 & 索引 & 外键\tabularnewline
\midrule
id & bigint & primary &\tabularnewline
name & varchar(255) & &\tabularnewline
course_name & varchar(255) & &\tabularnewline
description & text & &\tabularnewline
invite_code & varchar(255) & index &\tabularnewline
\bottomrule
\end{longtable}


\subsection{user_in_classes}



\begin{longtable}[]{@{}cccc@{}}
\toprule
字段名称 & 类型 & 索引 & 外键\tabularnewline
\midrule
id & bigint & primary &\tabularnewline
user_id & bigint & index & users(id)\tabularnewline
class_id & bigint & index & classes(id)\tabularnewline
\bottomrule
\end{longtable}


\subsection{user_managing_classes}



\begin{longtable}[]{@{}cccc@{}}
\toprule
字段名称 & 类型 & 索引 & 外键\tabularnewline
\midrule
id & bigint & primary &\tabularnewline
user_id & bigint & index & users(id)\tabularnewline
class_id & bigint & index & classes(id)\tabularnewline
\bottomrule
\end{longtable}


\subsection{grades}



\begin{longtable}[]{@{}cccc@{}}
\toprule
字段名称 & 类型 & 索引 & 外键\tabularnewline
\midrule
id & bigint & primary &\tabularnewline
total & bigint & &\tabularnewline
detail & JSON & &\tabularnewline
user_id & bigint & index & users(id)\tabularnewline
class_id & bigint & index & classes(id)\tabularnewline
\bottomrule
\end{longtable}


\subsection{problem_sets}



\begin{longtable}[]{@{}cccc@{}}
\toprule
字段名称 & 类型 & 索引 & 外键\tabularnewline
\midrule
id & bigint & primary &\tabularnewline
class_id & bigint & index & classes(id)\tabularnewline
name & varchar(255) & &\tabularnewline
description & text & &\tabularnewline
start_time & timestamp & &\tabularnewline
end_time & timestamp & &\tabularnewline
created_at & timestamp & &\tabularnewline
updated_at & timestamp & &\tabularnewline
\bottomrule
\end{longtable}


\subsection{problem_in_problem_sets}



\begin{longtable}[]{@{}cccc@{}}
\toprule
字段名称 & 类型 & 索引 & 外键\tabularnewline
\midrule
id & bigint & primary &\tabularnewline
problem_id & bigint & index & problems(id)\tabularnewline
problem_set_id & bigint & index & problem_sets(id)\tabularnewline
\bottomrule
\end{longtable}

\subsection{problems}



\begin{longtable}[]{@{}cccc@{}}
\toprule
字段名称 & 类型 & 索引 & 外键\tabularnewline
\midrule
id & bigint & primary &\tabularnewline
name & varchar(255) & &\tabularnewline
description & text & &\tabularnewline
attach_file_name & varchar(255) & &\tabularnewline
public & boolean & &\tabularnewline
memory_limit & bigint & &\tabularnewline
time_limit & bigint & &\tabularnewline
build_arg & varchar(2047) & &\tabularnewline
compare_script_name & text & &\tabularnewline
\bottomrule
\end{longtable}



% \chapter*{数据库设计}

\chapter{创建和删除数据库}
\newcommand{\pgdb}{postgres}
\section{实验目的}
本实验要求使用这两种方法SQL语句创建和删除数据库,实验目的在于:
\begin{outline}[enumerate]
    \1 学习使用SQL语句建立与管理数据库。
    \1 学会SQL语句的排错技术。
    \1 了解数据文件、日志文件等相关概念。
    \1 建立案例数据库以及自己设计的数据库,为以后的实验做准备。
    \1 对常见错误操作,进行测试,加深对数据库管理相关语句以及操作的理解。
\end{outline}
\section{实验步骤}
\subsection{新建数据库}
\subsubsection{查看当前数据库情况}
使用\sql{\l}命令查看当前数据库情况,运行结果如下所示:
\begin{run}
\l
\end{run}
\subsubsection{使用SQL语句创建数据库}
\begin{runsilent}
    drop database if exists dbexp;
\end{runsilent}
使用 \sql{create database} 命令创建数据库,运行结果如下所示:
\begin{run}
    create database dbexp;
\end{run}
\subsubsection{观察数据库变化}
使用\sql{\l}命令查看当前数据库情况,运行结果如下所示:
\begin{run}
\l
\end{run}
可以看到,增加了\texttt{dbexp}数据库。

\subsection{删除数据库}
\subsubsection{使用SQL语句删除数据库}
使用 \sql{drop database} 命令删除数据库,运行结果如下所示:
\begin{run}
    drop database dbexp;
\end{run}
\subsubsection{观察数据库变化}
使用\sql{\l}命令查看当前数据库情况,运行结果如下所示:
\begin{run}
\l
\end{run}
可以看到,删除了\texttt{dbexp}数据库。

\begin{runsilent}
    create database dbexp;
\end{runsilent}
\renewcommand{\pgdb}{dbexp}

\section{思考题}
\subsection*{数据库文件有哪些增长方式?}
\begin{outline}[enumerate]
    \1 按百分比增长(例如:每次增长10\%)。
    \1 按固定长度增长(例如:每次增长1MiB)。
\end{outline}
\subsection*{日志文件的作用是什么?}
记录数据库执行过的所有命令。可以根据日志文件诊断数据库或恢复数据库(如:当服务器意外断电时,可能数据库文件被破坏,此时可以用binlog来恢复数据库文件。)

\section{心得体会}

实验过程中,碰到的主要问题就是实验用的数据库用户(\texttt{dbtest})默认没有建立数据库的权限。需要执行 \sql|alter user dbtest CREATEDB;| 来授予权限。

\chapter{创建和删除基本表}
\section{实验目的}
本实验的学习目标在于熟练掌握数据库基本表的创建、修改和删除的方法,具体实验目的如下:
\begin{outline}[enumerate]
    \1 学会使用SQL语句创建、修改和删除表。
    \1 学会使用SQL语句设置常用的数据完整性约束,含主键约束、外键约束、空值约束、UNIQUE约束、默认值以及CHECK约束等。
    \1 学会使用系统存储过程查看基本表信息。
    \1 熟悉SQL的常用数据类型。
    \1 理解相关概念:基本表与三级结构、实体完整性、参照完整性、用户定义完整性、主键、外键、空值、默认值等。
    \1 建立案例数据库以及自己设计的数据库的相关基本表,为后面的实验做准备。
    \1 测试各种异常、错误情况,加深对表管理操作以及相关知识点的理解。
\end{outline}

\section{实验步骤}
\subsection{创建表}
\subsubsection{查询当前数据库情况}
使用 \sql{\dt} 命令查看当前数据库情况,运行结果如下所示:
\begin{run}
    \dt
\end{run}
\subsubsection{创建表}
使用 \sql|CREATE TABLE| 命令在默认的public schema中创建一个数据表,并增加主键约束:
\begin{run}
    CREATE TABLE "users" (
        "id" bigserial,
        "username" varchar(30) NOT NULL,
        "nickname" varchar(30) NOT NULL,
        "email" varchar(320) NOT NULL,
        "password" varchar(60) NOT NULL,
        "created_at" timestamptz NOT NULL,
        "updated_at" timestamptz NOT NULL,
        "deleted_at" timestamptz,
        PRIMARY KEY ("id"),
        UNIQUE ("username"),
        UNIQUE ("email")
    )
\end{run}
\subsubsection{查询当前数据库情况}
使用 \sql{\dt} 命令查看当前数据库情况,运行结果如下所示:
\begin{run}
    \dt
\end{run}
可以看到,新增了users表。
\subsection{修改表}
\subsubsection{查询当前数据表情况}
使用 \sql{select} 命令查看当前数据表情况,运行结果如下所示:

\begin{run}
    SELECT column_name as Name, data_type as Type, is_nullable as Nullable, column_default as Default FROM information_schema.columns WHERE table_schema = 'public' AND table_name = 'users';
\end{run}

\subsubsection{修改数据表}
使用 \sql{alter table} 命令修改数据表,运行结果如下所示:

\begin{run}
    alter table users add column "age" integer;
\end{run}

\subsubsection{查询修改后数据表情况}
使用 \sql{select} 命令查看修改后数据表情况,运行结果如下所示:

\begin{run}
    SELECT column_name as Name, data_type as Type, is_nullable as Nullable, column_default as Default FROM information_schema.columns WHERE table_schema = 'public' AND table_name = 'users';
\end{run}

可以看到,增加了age字段。

\subsection{删除表}
\subsubsection{查询当前数据库情况}
使用 \sql{\dt} 命令查看当前数据库情况,运行结果如下所示:

\begin{run}
    \dt
\end{run}

\subsubsection{删除数据表}
使用 \sql{drop} 命令删除数据表,运行结果如下所示:
\begin{run}
    drop table users;
\end{run}
\subsubsection{查询操作后数据库情况}
使用 \sql{\dt} 命令查看操作后数据库情况,运行结果如下所示:

\begin{run}
    \dt
\end{run}
可以看到,不存在任何数据表,删除成功。

\begin{runsilent}
    CREATE TABLE "users" (
        "id" bigserial,
        "username" varchar(30) NOT NULL,
        "nickname" varchar(30) NOT NULL,
        "email" varchar(320) NOT NULL,
        "password" varchar(60) NOT NULL,
        "created_at" timestamptz NOT NULL,
        "updated_at" timestamptz NOT NULL,
        "deleted_at" timestamptz,
        PRIMARY KEY ("id"),
        UNIQUE ("username"),
        UNIQUE ("email")
    )
\end{runsilent}

\subsection{创建外键约束}
创建班级表,并创建外键约束:
\begin{run}
    CREATE TABLE "classes" (
        "id" bigserial,
        "name" varchar(255) NOT NULL,
        "course_name" varchar(255) NOT NULL,
        "description" text DEFAULT '',
        "invite_code" varchar(255) NOT NULL DEFAULT '',
        "created_at" timestamptz,
        "updated_at" timestamptz,
        "deleted_at" timestamptz,
        PRIMARY KEY ("id")
    );

    CREATE TABLE "user_in_classes" (
        "class_id" bigint not null,
        "user_id" bigint not null,
        PRIMARY KEY ("class_id", "user_id"),
        CONSTRAINT "fk_user_in_classes_class" FOREIGN KEY ("class_id") REFERENCES "classes"("id"),
        CONSTRAINT "fk_user_in_classes_user" FOREIGN KEY ("user_id") REFERENCES "users"("id")
    );
\end{run}
\subsubsection{测试外键是否创建成功}
尝试插入一条违反外键约束的数据:
\begin{run}
    insert into user_in_classes (class_id, user_id) values (2, 2);
\end{run}
可以看到,系统阻止了非法数据插入,外键创建成功。
\subsubsection{创建非空和唯一约束}
已经在第一步中创建了非空和唯一约束。下面验证是否成功:
\begin{run}
    insert into users (username, nickname, email, password, created_at, updated_at, deleted_at) values (null, 'test', 'test@test.com', 'password', '2021-12-14 00:00:00', '2021-12-14 00:00:00', null);
    insert into users (username, nickname, email, password, created_at, updated_at, deleted_at) values ('test', 'test', 'test@test.com', 'password', '2021-12-14 00:00:00', '2021-12-14 00:00:00', null);
    insert into users (username, nickname, email, password, created_at, updated_at, deleted_at) values ('test', 'test', 'test@test.com', 'password', '2021-12-14 00:00:00', '2021-12-14 00:00:00', null);
    insert into users (username, nickname, email, password, created_at, updated_at, deleted_at) values ('test1', 'test', 'test@test.com', 'password', '2021-12-14 00:00:00', '2021-12-14 00:00:00', null);
\end{run}
\begin{runsilent}
    truncate table users cascade;
\end{runsilent}
可以看到,系统阻止了非法数据插入,非空和唯一约束创建成功。
\subsection{默认值}
使用 \sql{select} 命令查看当前数据表结构,运行结果如下所示:

\begin{run}
    SELECT column_name as Name, data_type as Type, is_nullable as Nullable, column_default as Default FROM information_schema.columns WHERE table_schema = 'public' AND table_name = 'users';
\end{run}

可以看到 \sql{id} 列的默认值为 \sql{users_id_seq} 序列的下一个值。

\subsection{check约束}
\begin{runsilent}
    truncate table users cascade;
\end{runsilent}
\begin{run}
    ALTER TABLE users ADD CONSTRAINT check_username_length CHECK (LENGTH(username) > 6);
\end{run}
\subsubsection{检查check约束}
尝试插入违反check约束的数据:
\begin{run}
    insert into users (username, nickname, email, password, created_at, updated_at, deleted_at) values ('testtest', 'testtest', 'test1@test.com', 'password', '2021-12-14 00:00:00', '2021-12-14 00:00:00', null);insert into users (username, nickname, email, password, created_at, updated_at, deleted_at) values ('test', 'test', 'test2@test.com', 'password', '2021-12-14 00:00:00', '2021-12-14 00:00:00', null);
\end{run}
可以看到运行失败,系统阻止了非法数据插入。
\begin{runsilent}
    truncate table users cascade;
    alter table users drop constraint check_username_length;
\end{runsilent}

\subsection{创建后续实验所需要的其他数据表}

\begin{run}
    CREATE TABLE "user_manage_classes" (
        "class_id" bigint,
        "user_id" bigint,
        PRIMARY KEY ("class_id", "user_id"),
        CONSTRAINT "fk_user_manage_classes_class" FOREIGN KEY ("class_id") REFERENCES "classes"("id"),
        CONSTRAINT "fk_user_manage_classes_user" FOREIGN KEY ("user_id") REFERENCES "users"("id")
    );
    
    CREATE TABLE "roles" (
        "id" bigserial,
        "name" text,
        "target" text,
        PRIMARY KEY ("id")
    );
    
    CREATE TABLE "permissions" (
        "id" bigserial,
        "role_id" bigint,
        "name" text,
        PRIMARY KEY ("id"),
        CONSTRAINT "fk_roles_permissions" FOREIGN KEY ("role_id") REFERENCES "roles"("id")
    );
    
    CREATE TABLE "tokens" (
        "id" bigserial,
        "token" text,
        "user_id" bigint,
        "remember_me" boolean,
        "created_at" timestamptz,
        "updated_at" timestamptz,
        PRIMARY KEY ("id"),
        CONSTRAINT "fk_tokens_user" FOREIGN KEY ("user_id") REFERENCES "users"("id")
    );
    
    CREATE TABLE "webauthn_credentials" (
        "id" bigserial,
        "user_id" bigint,
        "content" text,
        "created_at" timestamptz,
        PRIMARY KEY ("id"),
        CONSTRAINT "fk_users_credentials" FOREIGN KEY ("user_id") REFERENCES "users"("id")
    );
    
    CREATE TABLE "problem_sets" (
        "id" bigserial,
        "class_id" bigint NOT NULL,
        "name" varchar(255) NOT NULL,
        "description" text,
        "start_time" timestamptz,
        "end_time" timestamptz,
        "created_at" timestamptz,
        "updated_at" timestamptz,
        "deleted_at" timestamptz,
        PRIMARY KEY ("id"),
        CONSTRAINT "fk_classes_problem_sets" FOREIGN KEY ("class_id") REFERENCES "classes"("id"),
        CONSTRAINT "fk_problem_sets_class" FOREIGN KEY ("class_id") REFERENCES "classes"("id")
    );
    
    CREATE TABLE "grades" (
        "id" bigserial,
        "user_id" bigint,
        "problem_set_id" bigint,
        "class_id" bigint,
        "detail" JSON,
        "total" bigint,
        "created_at" timestamptz,
        "updated_at" timestamptz,
        PRIMARY KEY ("id"),
        CONSTRAINT "fk_grades_user" FOREIGN KEY ("user_id") REFERENCES "users"("id"),
        CONSTRAINT "fk_grades_problem_set" FOREIGN KEY ("problem_set_id") REFERENCES "problem_sets"("id"),
        CONSTRAINT "fk_grades_class" FOREIGN KEY ("class_id") REFERENCES "classes"("id"),
        CONSTRAINT "fk_problem_sets_grades" FOREIGN KEY ("problem_set_id") REFERENCES "problem_sets"("id"),
        CONSTRAINT "fk_users_grades" FOREIGN KEY ("user_id") REFERENCES "users"("id")
    );
    
    CREATE TABLE "scripts" (
        "name" text,
        "filename" text,
        "created_at" timestamptz,
        "updated_at" timestamptz,
        PRIMARY KEY ("name")
    );
    
    CREATE TABLE "problems" (
        "id" bigserial,
        "name" varchar(255) NOT NULL DEFAULT '',
        "description" text,
        "attachment_file_name" varchar(255) NOT NULL DEFAULT '',
        "public" boolean NOT NULL DEFAULT false,
        "privacy" boolean NOT NULL DEFAULT false,
        "memory_limit" bigint NOT NULL DEFAULT 0,
        "time_limit" bigint NOT NULL DEFAULT 0,
        "language_allowed" varchar(255) NOT NULL DEFAULT '',
        "build_arg" varchar(2047) NOT NULL DEFAULT '',
        "compare_script_name" text NOT NULL DEFAULT '0',
        "created_at" timestamptz,
        "updated_at" timestamptz,
        "deleted_at" timestamptz,
        PRIMARY KEY ("id"),
        CONSTRAINT "fk_problems_compare_script" FOREIGN KEY ("compare_script_name") REFERENCES "scripts"("name")
    );

    CREATE TABLE "problems_in_problem_sets" (
        "problem_set_id" bigint,
        "problem_id" bigint,
        PRIMARY KEY ("problem_set_id",
        "problem_id"),
        CONSTRAINT "fk_problems_in_problem_sets_problem_set" FOREIGN KEY ("problem_set_id") REFERENCES "problem_sets"("id"),
        CONSTRAINT "fk_problems_in_problem_sets_problem" FOREIGN KEY ("problem_id") REFERENCES "problems"("id")
    );

    CREATE TABLE "user_has_roles" (
        "id" bigserial,
        "user_id" bigint NOT NULL,
        "role_id" bigint NOT NULL,
        "target_id" bigint,
        CONSTRAINT "fk_user_has_roles_user" FOREIGN KEY ("user_id") REFERENCES "users"("id"),
        CONSTRAINT "fk_user_has_roles_role" FOREIGN KEY ("role_id") REFERENCES "roles"("id")
    );
    
\end{run}

\section{思考题}
\subsection*{什么叫做外键?}
如果公共关键字在一个关系中是主关键字,那么这个公共关键字被称为另一个关系的外键。
\subsection*{外键的作用是什么?}
当两个表中数据存在依赖关系时,保证数据一致性和完整性,并提供跨表查询的索引。
\section{心得体会}
在本次实验过程中,我了解了创建、删除、修改表的方法,并掌握了SQL的常用数据类型,了解了外键、唯一等约束的存在意义。在实际生产过程中,除了在应用端对数据进行检查之外,还应该尽可能把约束写入数据库中,保证数据一致性。

实验过程中碰到的主要问题就是OpenGauss 2.0版本不支持JSONB数据类型,只得使用JSON。

\chapter{数据的增删改}
\section{实验目的}
有关数据库中表的更新操作的实验,主要目的是:
\begin{outline}[enumerate]
    \1 学会使用SQL语句进行数据的增删改。
    \1 掌握数据增删改对数据约束的影响,深入理解主键约束、外键约束、check约束以及空值、默认值等相关概念。
    \1 熟练掌握各种数据类型的使用。
    \1 对于案例数据库以及自己设计的数据库中的基本表,插入数据,作为后面查询实验的基础
\end{outline}

\section{实验步骤}
\subsection{插入数据}
\subsubsection{使用 \sql{insert} 指令插入数据}
\begin{run}
    insert into users (username, nickname, email, password, created_at, updated_at, deleted_at) values 
    ('test1', 'test1', 'test1@test1.com', 'password', '2021-12-14 00:00:00', '2021-12-14 00:00:00', null),
    ('test2', 'test2', 'test2@test2.com', 'password', '2021-12-14 00:00:00', '2021-12-14 00:00:00', null),
    ('test3', 'test3', 'test3@test3.com', 'password', '2021-12-14 00:00:00', '2021-12-14 00:00:00', null),
    ('test4', 'test4', 'test4@test4.com', 'password', '2021-12-14 00:00:00', '2021-12-14 00:00:00', null);
\end{run}
\subsubsection{查看插入的结果}
\begin{run}
    select * from users;
\end{run}
\subsection{删除数据}
\subsubsection{使用 \sql{delete} 指令插入数据}
\begin{run}
    delete from users where username = 'test2';
\end{run}
\subsubsection{查看删除的结果}
\begin{run}
    select * from users;
\end{run}
可以看到,第二个用户被删除了。
\subsection{修改数据}
\subsubsection{使用 \sql{update} 指令修改数据}
\begin{run}
    update users set email = CONCAT('changed_', email) where true;
    update users set nickname = CONCAT(nickname, '_changed') where id = 10;
\end{run}
\subsubsection{查看修改的结果}
\begin{run}
    select * from users;
\end{run}
可以看到,邮箱和昵称字段的数据被修改了。
\section{思考题}
\subsection*{PostgreSQL和OpenGauss提供了哪些类型的约束?}
CHECK、NOT NULL、UNIQUE、PRIMARY KEY、FOREIGN KEY、EXCLUDE。

\subsection*{\sql{delete} 语句和 \sql{drop table} 语句有何不同?}
前者只删除表中部分或全部数据,后者在删除全部数据的同时会删掉表结构。

\section{心得体会}

本次实验过程中我熟悉了如何在数据表中进行增删改操作。



\chapter{数据的检索}

\begingroup
\renewcommand{\cleardoublepage}{}
\renewcommand{\clearpage}{}
\chapter*{单表查询}
\endgroup
\section{实验目的}

单表查询的实验是使用SELECT语句从单一基本表查询数据,主要目的是:
\begin{outline}[enumerate]
    \1 学会SELECT子句各种基本用法。
    \1 熟悉单表查询中各种WHERE条件的使用方法。
    \1 掌握常用的聚合函数的用法。
    \1 掌握分组统计的概念,熟悉GROUP BY 子句以及HAVING子句的基本用法。
    \1 掌握结果集输出时的各种排序方法,ORDER  BY子句的常用方法。
\end{outline}
\section{实验步骤}
\subsection{数据准备}
首先插入一些示例数据,用于以后查询。
\begin{run}
    insert into roles (name, target) values 
        ('admin', null),
        ('creator', 'problem'),
        ('creator', 'class'),
        ('manager', 'problem'),
        ('student', 'class')
    ;
    insert into permissions (role_id, name) values 
        (1, 'all'),
        (2, 'all'),
        (3, 'all'),
        (4, 'read'),
        (4, 'change'),
        (4, 'update')
    ;
    update users set deleted_at = '2021-12-14 00:00:00' where id = 10;
    insert into user_has_roles (user_id, role_id, target_id) values 
        (7, 1, null),
        (7, 2, 3),
        (7, 4, 4),
        (9, 2, 4),
        (9, 5, null)
    ;
\end{run}
\subsection{查询语句的使用}
下面,通过EduOJ的真实使用场景来展示不同查询语句的使用。
\subsubsection{都有哪些role具有permission?}
\begin{run}
    select role_id from permissions group by role_id;
    select distinct role_id from permissions;
\end{run}
\subsubsection{有哪些role具有2个以上的permission?}
\begin{run}
    select role_id from permissions group by role_id having count(*) >= 2;
\end{run}
\subsubsection{按照username排序,第3个用户是哪个用户?}
\begin{run}
    select * from users order by username asc offset 2 limit 1;
\end{run}
\subsubsection{没被删除的用户有哪些?}
\begin{run}
    select * from users where deleted_at is null;
\end{run}
\subsubsection{被删除了的用户有哪些?}
\begin{run}
    select * from users where deleted_at is not null;
\end{run}

\section{思考题}
\subsection*{什么是空值?}
空值是表示该行没有值的一种特殊状态,而不是值为空(如空字符串)。
\subsection*{为什么空值不用等号判定?}
因为空值不是值,而是一种特殊状态。如:有\sql{UNIQUE}约束的表中可以有多个NULL。
\subsection*{聚合函数可以出现在什么字句中?}
\sql{SELECT}和\sql{HAVING}。
\subsection*{什么情况下使用\sql{HAVING}?}
当需要对 \sql{group by} 后的数据进行进一步筛选时。

筛选顺序: \sql{where -> group by -> having}


\chapter*{多表查询}
\setcounter{section}{0}
\section{实验目的}

多表查询的实验是使用查询语句从多个基本表或视图查询数据,包含连接查询(内连接)、集合查询以及子查询3种查询方法,本实验主要目的是:
\begin{outline}[enumerate]
    \1 学会内连接查询的表示方法(标准表示法或简约表示法均可),以及自连接的表示法。
    \1 学会集合查询的达,包括UNION、INTERSECT和EXCEPT的表达,集合运算的“并兼容”问题。
    \1 学会子查询即嵌套查询的使用方法,包括3种形式引入子查询的方法:[NOT] IN、 比较运算符与ALL|ANY 和EXISTS;理解相关子查询和独立子查询的概念,学会相关子查询的表达方法。
    \1 学会上述3种多表查询方法的综合应用。
    \1 学会上述3种多表查询与GROUP BY 子句以及ORDER BY 子句的联合使用。
    \1 深入理解主键、外键的概念。
    \1 深入理解实体完整性约束与参照完整性约束的概念。
\end{outline}
学习使用SELECT语句在多张基本表中查询各类信息。熟悉WHERE条件的表达、DISTINCT的使用、连接条件与选择条件的表达。理解连接运算。


\section{实验步骤}
\subsection{多表查询语句的使用}
下面,结合EduOJ的真实使用场景,展示多表查询语句的使用。
\subsubsection*{具有针对id为3的problem的all权限的用户有哪些?}
\begin{run}
    select * from users where id in (
        select user_id from user_has_roles where role_id in (
            select r.id from roles r 
            inner join permissions p on r.id = p.role_id 
            where p.name = 'all' and r.target = 'problem'
        ) and target_id = 3
    );
\end{run}
\subsubsection*{具有针对id为4的problem的read或all权限的用户有哪些?}
\begin{run}
    select * from users where id in (
        select user_id from user_has_roles where role_id in (
            select r.id from roles r 
            inner join permissions p on r.id = p.role_id 
            where p.name in ('all', 'read') and r.target = 'problem'
        ) and target_id = 4
    );
\end{run}
\begin{run}
    select * from users where id in (
        select user_id from user_has_roles where role_id in (
            select r.id from roles r 
            inner join permissions p on r.id = p.role_id 
            where p.name = 'all' and r.target = 'problem'
            union
            select r.id from roles r 
            inner join permissions p on r.id = p.role_id 
            where p.name = 'read' and r.target = 'problem'
        ) and target_id = 4
    );
\end{run}
\subsubsection*{具有针对id为4的problem的read或all权限的第二个用户是哪个?}
\begin{run}
    select * from users where id in (
        select user_id from user_has_roles where role_id in (
            select r.id from roles r 
            inner join permissions p on r.id = p.role_id 
            where p.name in ('all', 'read') and r.target = 'problem'
        ) and target_id = 4
    ) order by id asc offset 1 limit 1;
\end{run}
\subsubsection*{同时具有针对id为3的problem的all或read权限以及id为4的problem的all或read权限的用户有哪些?}
\begin{run}
    select * from users where id in (
        select user_id from user_has_roles where role_id in (
            select r.id from roles r 
            inner join permissions p on r.id = p.role_id 
            where p.name in ('all', 'read') and r.target = 'problem'
        ) and target_id = 4
        intersect
        select user_id from user_has_roles where role_id in (
            select r.id from roles r 
            inner join permissions p on r.id = p.role_id 
            where p.name in ('all', 'read') and r.target = 'problem'
        ) and target_id = 3
    );
\end{run}
\subsubsection*{具有针对id为4的problem的all或read权限但没有id为3的problem的all或read权限的用户有哪些?}
\begin{run}
    select * from users where id in (
        select user_id from user_has_roles where role_id in (
            select r.id from roles r 
            inner join permissions p on r.id = p.role_id 
            where p.name in ('all', 'read') and r.target = 'problem'
        ) and target_id = 4
        except
        select user_id from user_has_roles where role_id in (
            select r.id from roles r 
            inner join permissions p on r.id = p.role_id 
            where p.name in ('all', 'read') and r.target = 'problem'
        ) and target_id = 3
    );
\end{run}
\subsubsection*{具有针对id为4的problem的read、change、update这3个权限中至少2个的用户有哪些?}
\begin{run}
    select users.* from users inner join (
        select ur.user_id from user_has_roles ur
        inner join permissions p on ur.role_id = p.role_id
        where ur.target_id = 4 group by user_id having count(*) >= 2
    ) as rr on users.id = rr.user_id
\end{run}
\section{思考题}
\subsection*{连接条件一定是对应属性相等吗?}
不一定,可以是$\ge$等,甚至是 \sql{true}。
\subsection*{所有的查询都可以使用多表连接和子查询两种方法吗?}
不一定。\sql{join} 的条件写 \sql{true} 可以做笛卡尔乘积,但是无法用子查询达到一样的效果。

在绝大部分情况下,二者可以互相替换。二者的区别在于,子查询是『逻辑上』更合理的方式,而连接则可以更有效的运用表间外键的索引,达到更高的效率。

\chapter{创建和删除视图}
\section{实验目的}
本实验主要是通过学习视图的相关知识,了解数据库对象——视图的作用,创建、修改、删除视图及视图加密等相关技术。具体要求如下:
\begin{outline}[enumerate]
    \1 掌握视图的基本概念,了解视图在数据库系统中的作用及原理。
    \1 掌握使用-SL进行视图的创建、修改和删除操作。
    \1 了解基于视图进行表数据的修改及其注意事项。
    \1 了解视图加密的方法。
\end{outline}
\section{实验步骤}
\subsection{视图的创建}
\begin{run}
    create view undeleted_users as select * from users where deleted_at is not null;
\end{run}
\subsection{视图减少一列}
\begin{run}
    create view undeleted_users_no_deleted_at as select id, username, nickname, email, password, created_at, updated_at from undeleted_users where true;
\end{run}
\subsection{插入一条记录}
\begin{run}
    insert into undeleted_users(username, nickname, email, password,  created_at, updated_at)
    values ('test0', 'test0', 'test0@test0.com', 'password',  '2021-12-14 00:00:00', '2021-12-14 00:00:00');
\end{run}

注意,OpenGauss不支持对于没有Trigger的视图的插入操作。

\subsection{删除一条记录}
\begin{run}
    delete from undeleted_users where username = 'test4';
\end{run}

\subsection{修改一条记录}
\begin{run}
    insert into undeleted_users(username, nickname, email, password,  created_at, updated_at, deleted_at)
    values ('test4', 'test4', 'test4@test4.com', 'password',  '2021-12-14 00:00:00', '2021-12-14 00:00:00', '2021-12-14 00:00:00');
    update undeleted_users set username = 'undeleted_users' where undeleted_users = 'test4';
\end{run}

\subsection{限制引用表的删除}
\begin{run}
    CREATE RULE do_not_delete_user AS ON DELETE TO users DO INSTEAD NOTHING;
\end{run}
测试删除:
\begin{run}
    delete from undeleted_users where username = 'test4';
\end{run}
重新查询:
\begin{run}
    select * from undeleted_users;
\end{run}
可以看到,删除操作没有成功。
\begin{runsilent}
    drop rule do_not_delete_user on users;
\end{runsilent}
\section{思考题}
\subsection*{视图和基本表有何不同?}
视图是编译过的 \sql{select} 语句。如果视图是非持久话的,内存中不会存在一个视图的表结构。如果视图是持久化的,那么会在创建视图的时候创建临时表储存结果,之后只能手动刷新。此时,持久化视图创建的临时表除去可以手动刷新之外,表现和基本表就是一样的。

\chapter{创建和删除索引}
\section{实验目的}

本实验主要目的在于通过学习数据库索引的相关知识,了解数据库索引的结构、类型,创建方法以及索引的基本维护方法(重新生成索引和重新组织索引)。具体要求如下:
\begin{outline}
    \1 掌握数据库索引基本概念,以及索引的基本类型。
    \1 学会使用SQL创建、查看和修改索引。
    \1 学会使用SQL重新生成索引。
    \1 学会使用SQL重新组织索引。
\end{outline}
\section{实验步骤}
\subsection{加索引前,分析SQL语句执行时间}
以下列一句为例,分析加索引前所需要的执行时间:
\begin{run}
    explain select * from users where id in (
        select user_id from user_has_roles where role_id in (
            select r.id from roles r 
            inner join permissions p on r.id = p.role_id 
            where p.name = 'all' and r.target = 'problem'
        ) and target_id = 3
    );
\end{run}
可以发现,这个查询语句用了4层循环,其中1层优化为了join操作,整体cost为79.44。

\subsection{添加索引}

\begin{run}
    create index p_role_id on permissions (role_id);
    create index p_name on permissions (name);
    create index r_target on roles (target);
    create index u_user_id on user_has_roles (user_id);
    create index u_target_id on user_has_roles (target_id);
    create index u_role_id on user_has_roles (role_id);
    create index u_nickname on users (nickname);
    create index u_email on users (email);
    create index u_username on users (username);
    create index c_deleted_at on classes(deleted_at);
    create index w_user_id on webauthn_credentials(user_id);
    create index c_invite_code on classes(invite_code);
    create index uc_user_id on user_in_classes(user_id);
    create index uc_class_id on user_in_classes(class_id);
    create index ucm_user_id on user_manage_classes(user_id);
    create index ucm_class_id on user_manage_classes(class_id);
    create index g_user_id on grades(user_id);
    create index g_class_id on grades(class_id);
    create index ps_class_id on problem_sets(class_id);
    create index pp_problem_id on problems_in_problem_sets(problem_id);
    create index pp_problem_set_id on problems_in_problem_sets(problem_set_id);
    reindex database dbexp;
\end{run}

\subsection{加索引后,分析SQL语句执行时间}
以下列一句为例,分析加索引后所需要的执行时间:
\begin{run}
    explain select * from users where id in (
        select user_id from user_has_roles where role_id in (
            select r.id from roles r 
            inner join permissions p on r.id = p.role_id 
            where p.name = 'all' and r.target = 'problem'
        ) and target_id = 3
    );
\end{run}
可以看到,SQL语句准备的时间由70.83降低到了0,执行全部结果的时间由75.53降低到了4.33。

同时可以发现,PostgreSQL优化SQL查询的能力高于OpenGauss。

\subsection{删除索引}
\begin{run}
    drop index p_role_id;
\end{run}
\begin{runsilent}
    analyse;
\end{runsilent}
\subsection{删除一条索引后,分析SQL执行时间}
\begin{run}
    explain select * from users where id in (
        select user_id from user_has_roles where role_id in (
            select r.id from roles r 
            inner join permissions p on r.id = p.role_id 
            where p.name = 'all' and r.target = 'problem'
        ) and target_id = 3
    );
\end{run}
可以看到语句执行变慢,索引删除成功。

\section{思考题}
\subsection*{索引在数据库中的作用是什么?}
加快查询速度,也可以保证数据唯一。
\subsection*{索引有哪几种类型?}
B-tree、Hash、GiST和GIN。

\end{document}
