\documentclass{ctexart}
\usepackage[T1]{fontenc}
\usepackage[a4paper,top=1.5cm,bottom=1.5cm,left=2cm,right=2cm,marginparwidth=1.75cm]{geometry}
\usepackage{mathtools}
\usepackage{tikz}
\usepackage{booktabs}
\usepackage{caption}
\usepackage{outlines}
\usepackage{graphicx}
\usepackage{amsthm}
\usepackage[colorlinks=false, allcolors=blue]{hyperref}
\renewcommand{\tableautorefname}{表}
\DeclarePairedDelimiter{\set}{\{}{\}}
\DeclarePairedDelimiter{\paren}{(}{)}
\graphicspath{ {./images/} }\title{数据库第四章作业}
\author{卢雨轩 19071125}
% \date{\today}
\ctexset{
    section = {
        titleformat = \raggedright,
        name = {,},
        number = \chinese{section}、
    },
    paragraph = {
        runin = false
    },
    today = small,
    figurename = 图,
    contentsname = 目录,
    tablename = 表,
}
\begin{document}
\maketitle
\begin{outline}[enumerate]
    \1 设有关系模式R(U,F)。其中,$U=\set{A, B, C, D},F=\set{AB→C,  C→D,  D→A}$, 求R的所有键码。

        B只出现在左边。
        \begin{equation*}
            \begin{cases}
            AB: AB \to C, C \to D \\
            BC: C \to D, D \to A \\
            BD: D \to A, AB \to C \\
            \end{cases}
        \end{equation*}
        所以$AB,BC,BD$均为键码。
    \1 已知两个关系模式$R1(\set{A,B,C,E}, \set{A→B,C→E})$和$R2(\set{A,C,D}, \set{(A,C)→D})$。问:在函数依赖范围内,R1和R2分别是第几范式的?
        \2 R1

            R1的键码:$AC$
            \3 1NF:符合。属性原子。
            \3 2NF:符合。无部分依赖。
            \3 3NF:复合。无传递依赖。
            \3 BCNF:$A \to B$,B不是超码。不符合。

            所以是3NF。
        \2 R2

            R2的键码:$AC$。


            \3 1NF:符合。属性原子。
            \3 2NF:符合。无部分依赖。
            \3 3NF:复合。无传递依赖。
            \3 BCNF:$AC \to D$,AC是超码,复合。

            所以是BCNF。
    \1 设有关系模式$R(U,F)$。其中,$U=\set{ A, B, C,D,E },F=\set{ D→E,  E→A, BD→C }$,且已知R的键码是$(B,D)$。请设计R的一个子模式,它把R无损连接性地分解到BCNF。(写出计算过程)。
        \2 $\rho = \set{R(U,F)}$
        \2 $D \to E$不符合BCNF。
            \3 $D_F^+ = \set{D,E,A}$
            \3 $U \to \begin{cases}
                U_1 = \set{A,D,E} \\
                U_2 = \set{B,C,D}
            \end{cases}$
            \3 $\rho = \begin{cases}
                R_1(\set{A,D,E}, \set{D \to E, E \to A}) \\
                R_2(\set{B,C,D}, \set{BD \to C})
            \end{cases}$
            \3 $R_1$不符合BCNF。
        \2 $E \to A$ 不符合BCNF。
            \3 $E_{F_1}^+ = \set{E, A}$
            \3 $U \to \begin{cases}
                U_1 = \set{E,A} \\
                U_2 = \set{D,E} \\
            \end{cases}$
            \3 $\rho = \begin{cases}
                R_{11}(\set{A,E}, \set{E \to A}) \\
                R_{12}(\set{D,E}, \set{D \to E}) \\
                R_2(\set{B,C,D}, \set{BD \to C})
            \end{cases}$
    \1 设有关系模式$R(U,F)$。其中,$U=\set{ A, B, C,D,E },    F=\set{ A→B,  B→A, B→C,A→C, C→A , D→E }$, 请设计R的一个子模式ρ ,它把R保持函数依赖地分解到3NF(写出计算过程)。
        \2 $F = \set{ A→B,  B→A, B→C,A→C, C→A , D→E }$
        \2 $F_m = \set{ A→B,  B→C, C→A , D→E }$
        \2 $U_0' = \emptyset$
        \2 $\begin{cases}
            F_1 = \set{A \to B} \\
            F_2 = \set{B \to C} \\
            F_3 = \set{C \to A} \\
            F_4 = \set{D \to E} \\
        \end{cases}$
        \2 $\begin{cases}
            R_1(\set{A,B}, \set{A \to B}) \\
            R_2(\set{B,C}, \set{B \to C}) \\
            R_3(\set{A,C}, \set{C \to A}) \\
            R_4(\set{D,E}, \set{D \to E}) \\
        \end{cases}$
\end{outline}
\end{document}
