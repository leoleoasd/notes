\markdownRendererHeadingOne{数据需求描述}\markdownRendererInterblockSeparator
{}disclaimer: 本文档基于\markdownRendererLink{EduOJ}{https://github.com/EduOJ/backend}{https://github.com/EduOJ/backend}{}的数据库设计中的用户、班级、作业、权限管理部分。\markdownRendererInterblockSeparator
{}\markdownRendererHeadingTwo{管理员:}\markdownRendererInterblockSeparator
{}\markdownRendererUlBeginTight
\markdownRendererUlItem 用户增删改查\markdownRendererUlItemEnd 
\markdownRendererUlItem 班级增删改查\markdownRendererUlItemEnd 
\markdownRendererUlItem 作业增删改查\markdownRendererUlItemEnd 
\markdownRendererUlItem 成绩查询、修改\markdownRendererUlItemEnd 
\markdownRendererUlEndTight \markdownRendererInterblockSeparator
{}\markdownRendererHeadingThree{用户相关}\markdownRendererInterblockSeparator
{}\markdownRendererUlBeginTight
\markdownRendererUlItem 姓名,学号,密码,邮箱\markdownRendererUlItemEnd 
\markdownRendererUlItem 权限相关:\markdownRendererInterblockSeparator
{}\markdownRendererUlBeginTight
\markdownRendererUlItem 『全局权限』(如:某个用户有权限创建用户、修改用户、创建班级、创建题目)\markdownRendererUlItemEnd 
\markdownRendererUlItem 『针对权限』(如:某个用户有针对id为5的班级的添加学生权限)\markdownRendererUlItemEnd 
\markdownRendererUlItem 希望能够『批量管理』:把权限授予给『角色』,让『用户』拥有『角色』。\markdownRendererUlItemEnd 
\markdownRendererUlEndTight \markdownRendererUlItemEnd 
\markdownRendererUlItem 统计信息,加快查询速度\markdownRendererUlItemEnd 
\markdownRendererUlEndTight \markdownRendererInterblockSeparator
{}\markdownRendererHeadingThree{班级相关}\markdownRendererInterblockSeparator
{}\markdownRendererUlBeginTight
\markdownRendererUlItem 课程名称,课堂名称,管理老师,课程描述\markdownRendererUlItemEnd 
\markdownRendererUlItem 用户、作业\markdownRendererInterblockSeparator
{}\markdownRendererUlBeginTight
\markdownRendererUlItem 和用户是多对多关系\markdownRendererUlItemEnd 
\markdownRendererUlItem 和作业是多对多关系\markdownRendererUlItemEnd 
\markdownRendererUlEndTight \markdownRendererUlItemEnd 
\markdownRendererUlEndTight \markdownRendererInterblockSeparator
{}\markdownRendererHeadingThree{作业相关}\markdownRendererInterblockSeparator
{}\markdownRendererUlBeginTight
\markdownRendererUlItem 标题,起止日期\markdownRendererUlItemEnd 
\markdownRendererUlItem 多道题目\markdownRendererUlItemEnd 
\markdownRendererUlItem 统计成绩\markdownRendererUlItemEnd 
\markdownRendererUlEndTight \markdownRendererInterblockSeparator
{}\markdownRendererHeadingThree{成绩查询、修改}\markdownRendererInterblockSeparator
{}\markdownRendererUlBeginTight
\markdownRendererUlItem 用户、班级、作业、成绩\markdownRendererUlItemEnd 
\markdownRendererUlItem 根据用户做题记录生成,用于加快计算,\markdownRendererStrongEmphasis{有数据冗余}。\markdownRendererUlItemEnd 
\markdownRendererUlEndTight \markdownRendererInterblockSeparator
{}\markdownRendererHeadingOne{数据库设计}\markdownRendererInterblockSeparator
{}\markdownRendererHeadingTwo{ER图}\markdownRendererInterblockSeparator
{}\markdownRendererImage{EduOJ用户、班级、作业、权限管理部分ER图}{./course-design-er.pdf}{./course-design-er.pdf}{}\markdownRendererInterblockSeparator
{}\markdownRendererHeadingTwo{关系模式}\markdownRendererInterblockSeparator
{}\markdownRendererUlBeginTight
\markdownRendererUlItem user(\underline{id}, nickname, username, email, password, email\_verified, created\_at, updated\_at)\markdownRendererUlItemEnd 
\markdownRendererUlItem roles(\underline{id}, name, target)\markdownRendererUlItemEnd 
\markdownRendererUlItem user\_has\_roles(\underline{id}, user\_id, )\markdownRendererUlItemEnd 
\markdownRendererUlItem permissions(\underline{id},role\_id, name)\markdownRendererUlItemEnd 
\markdownRendererUlItem tokens(\underline{id},user\_id,token, last\_used)\markdownRendererUlItemEnd 
\markdownRendererUlItem webauthn\_credentials(\underline{id}, user\_id, token, last\_used)\markdownRendererUlItemEnd 
\markdownRendererUlItem classes(\underline{id}, name, course\_name, description, invite\_code)\markdownRendererUlItemEnd 
\markdownRendererUlItem user\_in\_classes(\underline{id}, user\_id, class\_id)\markdownRendererUlItemEnd 
\markdownRendererUlItem user\_managing\_classes(\underline{id}, user\_id, class\_id)\markdownRendererUlItemEnd 
\markdownRendererUlItem grades(\underline{id}, total, detail, user\_id, class\_id, problem\_set\_id)\markdownRendererUlItemEnd 
\markdownRendererUlItem problem\_sets(\underline{id}, name, description, start\_time,end\_time)\markdownRendererUlItemEnd 
\markdownRendererUlItem problem\_in\_problem\_sets(\underline{id}, problem\_id, problem\_set\_id)\markdownRendererUlItemEnd 
\markdownRendererUlItem problems(\underline{id}, name, description, attach\_file\_name, public, memory\_limit, time\_limit, compare\_script\_name, build\_arg)\markdownRendererUlItemEnd 
\markdownRendererUlEndTight \markdownRendererInterblockSeparator
{}\markdownRendererHeadingTwo{范式判断}\markdownRendererInterblockSeparator
{}\markdownRendererHeadingThree{1NF}\markdownRendererInterblockSeparator
{}所有关系模式中,属性均是原子的,符合范式。\markdownRendererInterblockSeparator
{}\markdownRendererHeadingThree{2NF}\markdownRendererInterblockSeparator
{}所有关系模式中均依赖主键\underline{id},符合范式。\markdownRendererInterblockSeparator
{}\markdownRendererHeadingThree{3NF}\markdownRendererInterblockSeparator
{}表中除了主键之外所有属性均不互相依赖,符合范式。\markdownRendererInterblockSeparator
{}\markdownRendererHeadingThree{BCNF}\markdownRendererInterblockSeparator
{}所有关系模式均只有一个主属性,不存在其他键码,同时非主属性也依赖与键码,所以符合范式。\markdownRendererInterblockSeparator
{}\markdownRendererHeadingThree{4NF}\markdownRendererInterblockSeparator
{}所有关系模式均不存在平凡多值依赖,故符合4NF。\markdownRendererInterblockSeparator
{}\markdownRendererHeadingOne{数据表设计}\markdownRendererInterblockSeparator
{}\markdownRendererHeadingTwo{users}\markdownRendererInterblockSeparator
{}\markdownRendererTable{}{2}{4}{cccc}%
{{字段名称}%
{类型}%
{索引}%
{外键}%
}%
{{id}%
{bigint}%
{primary}%
{}%
}%
\relax