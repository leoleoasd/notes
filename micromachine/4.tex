\documentclass{ctexart}
\usepackage[T1]{fontenc}
\usepackage[a4paper,top=1.5cm,bottom=1.5cm,left=2cm,right=2cm,marginparwidth=1.75cm]{geometry}
\usepackage{mathtools}
\usepackage{tikz}
\usepackage{booktabs}
\usepackage{caption}
\usepackage{outlines}
\usepackage{graphicx}
\usepackage{amsthm}
\usepackage{minted}
\usepackage[colorlinks=false, allcolors=blue]{hyperref}
\renewcommand{\tableautorefname}{表}
\DeclarePairedDelimiter{\set}{\{}{\}}
\DeclarePairedDelimiter{\paren}{(}{)}
\graphicspath{ {./images/} }

\title{第四次微机接口作业}
\author{卢雨轩 19071125}
% \date{\today}
\ctexset{
    section = {
        titleformat = \raggedright,
        name = {,},
        number = \chinese{section}、
    },
    paragraph = {
        runin = false
    },
    today = small,
    figurename = 图,
    contentsname = 目录,
    tablename = 表,
}

\begin{document}

\maketitle

\begin{outline}[enumerate]
    \1 若某一终端以2400波特的速率发送异步串行数据,发送1位需要多少时间?假如一个字符包含7个数据位、1个奇偶校验位、1个停止位,发送1个字符需要多少时间?

        \2 需要 $1 / 2400 = 0.42 ms$。
        \2 需要 $1 / 2400 \times (7 + 1 + 1 + 1) = 4.2ms$
    \1 若8251A以9600波特的速率发送数据,波特率因子为16,发送时钟TxC频率为多少?
        \2 153600
    \1 若8251A的端口地址为FF0H,FF2H,要求8251A工作于异步工作方式,波特率因子为16,有7个数据位,1个奇校验位,1个停止位,试对8251A进行初始化编程。
\begin{minted}{gas}
MOV DX,FF2H
MOV AL,5AH
OUT DX,AL
MOV AL,00010101B
out DX, AL
\end{minted}

\1 某微机系统用串行方式接收外设送来的数据,再把数据送到CRT去显示,若波特率为1200,波特率因子为16,用8253产生收发时钟,系统时钟频率为5MHz,收发数据个数为COUNT,数据存放到数据段中以BUFFER为始址的内存单元中。8253和8251A的基地址分别为300H和304H。
\2 画出系统硬件连线图。
\2 编写8253和8251A的初始化程序。
\2 编写接收数据和发送数据的程序
\begin{minted}{gas}
mov dx, 303h
mov al, 110100b
out dx, al
mov dxm 300h
mov al, 260
out dx, al
mov al, 0
out dx, al
mov dx, 305h
mov al, 7ah
out dx, al
mov al, 15h 
out dx, al

L0: mov dx, 305h
L1: in al, dx
test al, 02h
jz l1
test al, 38h
jnz err
mov dx, 304h
in al, dx
mov bl, al
jmp L0
err: jmp err
\end{minted}
\1 设8255A的A口,B口,C口和控制字寄存器的端口地址分别80H,82H,84H和86H。要求A口工作在方式0输出,B口工作在方式0输入,C口高4位输入,低4位输出,试编写8255A的初始化程序。
\begin{minted}{gas}
mov al, 10001010b
out 086h, al
\end{minted}
\1 8255A的端口地址同第1题,要求PC4输出高电平,PC5输出低电平,PC6输出一个正脉冲,试写出完成这些功能的指令序列。
\begin{minted}{gas}
; PC4高电平
mov al, 1001b
out 86h, al
; PC5低电平
mov al, 1010b
out 86h, al

; PC6正脉冲
mov al, 1101b
out 86h, al
mov al, 1100b
out 86h, al
mov al, 1101b
out 86h, al
\end{minted}

\end{outline}
\end{document}
