\documentclass[zihao=5,linespread=1,heading=false,autoindent=0pt]{ctexart}
\usepackage[T1]{fontenc}
\usepackage[a4paper,top=1cm,bottom=1cm,left=1cm,right=1cm,marginparwidth=1.75cm]{geometry}
\usepackage{mathtools}
\usepackage{tikz}
\usepackage{booktabs}
\usepackage{caption}
\usepackage{outlines}
\usepackage{graphicx}
\usepackage{amssymb}
\usepackage{float}
\usepackage{amsthm}
\usepackage{enumitem}
\usepackage{titlesec}
\usepackage{wrapfig}
\usepackage{cancel}
\usepackage{multicol}
\usepackage{bm}
% \usepackage{breqn}
\usepackage[colorlinks=false, allcolors=blue]{hyperref}
% \usepackage{unicode-math}
% \setmathfont{texgyrepagella-math.otf}
\renewcommand{\tableautorefname}{表}
\DeclarePairedDelimiter{\set}{\{}{\}}
\DeclarePairedDelimiter{\paren}{(}{)}
\graphicspath{ {./images/} }

\makeatletter
\DeclareRobustCommand{\em}{%
  \@nomath\em \if b\expandafter\@car\f@series\@nil
  \normalfont \else \heiti\bfseries \fi}
\makeatother

\usetikzlibrary{automata,positioning}
\newcommand{\hl}[1]{\colorbox{yellow}{#1}}
\newcommand{\tto}{\Rightarrow}
\pagestyle{empty}
\newenvironment{citemize}%
{\begin{itemize}[parsep=0pt,itemsep=0pt,topsep=0pt,partopsep=0pt,labelwidth=1em,leftmargin=*]}
{\end{itemize}}
\newenvironment{cenumerate}%
{\begin{enumerate}[parsep=0pt,itemsep=0pt,topsep=0pt,partopsep=0pt,labelwidth=1em,leftmargin=*]}
{\end{enumerate}}
\linespread{1}
\setlength{\parindent}{0pt}
\setlength{\parskip}{0pt}
\setlength{\baselineskip}{0pt}
\setlength{\abovedisplayskip}{0pt}
\setlength{\belowdisplayskip}{0pt}
\setlength{\abovedisplayshortskip}{0pt}
\setlength{\belowdisplayshortskip}{0pt}
\titlespacing*{\section}{0pt}{0pt}{0pt}
\titlespacing*{\subsection}{0pt}{0pt}{0pt}
\titlespacing*{\subsubsection}{0pt}{0pt}{0pt}
\titlespacing*{\paragraph}{0pt}{0pt}{0pt}

\setlength{\multicolsep}{0.0pt}% 50% of original values
\setlength{\floatsep}{0pt plus 2pt minus 2pt}
\setlength{\textfloatsep}{0pt plus 2pt minus 2pt}
\setlength{\intextsep}{0pt plus 2pt minus 2pt}
\setlength{\medskipamount}{2pt}

\newcommand{\HRule}[1][\medskipamount]{\par
  \vspace*{\dimexpr-\parskip-\baselineskip+#1}
  \noindent\rule{\linewidth}{0.2mm}\par
  \vspace*{\dimexpr-\parskip-.5\baselineskip+#1}}

\newcommand{\includedrawio}[2][]{
    \immediate\write18{echo open -a /Applications/draw.io.app --args `pwd`/#2 --crop -x -o `pwd`/#2.pdf 2>&1 > t.out}
    % \immediate\write18{/Applications/draw.io.app/Contents/MacOS/draw.io #2 --crop -x -o #2.pdf}
    \includegraphics[#1]{#2.pdf}
}

\ctexset{
    section = {
        % titleformat = \raggedright,
        runin = true,
        format += \small,
        name = {,},
    },
    subsection/format += \small,
    subsubsection/format += \small,
    paragraph = {
        runin = false
    },
    today = small,
    figurename = 图,
    contentsname = 目录,
    tablename = 表,
}

\newtheoremstyle{exampstyle}
  {0pt} % Space above
  {0pt} % Space below
  {} % Body font
  {} % Indent amount
  {\bfseries} % Theorem head font
  {} % Punctuation after theorem head
  {2ex} % Space after theorem head
  {} % Theorem head spec (can be left empty, meaning `normal')

\theoremstyle{exampstyle} \newtheorem{definition}{定义}[section]
\theoremstyle{exampstyle} \newtheorem{example}{例}[section]
\theoremstyle{exampstyle} \newtheorem{theorem}{定理}[section]
\theoremstyle{exampstyle} \newtheorem{lemma}{引理}[section]
\theoremstyle{exampstyle} \newtheorem{myproof}{证明}[section]

\begin{document}

\begin{multicols*}{3}
\small
\paragraph{ PPT1(1.X) 微型计算机概述}
\begin{citemize}
     
    \item
      \textbf{1.1 微型计算机的特点和发展 1-14}
    \item
      \textbf{1.2 微型计算机的分类 1-17}
    
      \begin{citemize}
       
      \item
        1、按规模分类 1-17
      \item
        2、按微处理器的字长分类 1-18
      \end{citemize}
    \item
      \textbf{1.3 微处理器、微型计算机和微型计算机系统 1-19}
    
      \begin{citemize}
       
      \item
        1.3.1 微处理器CPU 1-20
      \item
        1.3.2 微型计算机 1-22
    
        \begin{citemize}
         
        \item
          微处理器 1-22
        \item
          存储器 1-22
        \item
          输入/输出接口电路 1-22
        \item
          系统总线 1-23
        \item
          数据总线、地址总线和控制总线 1-24
        \end{citemize}
      \item
        1.3.3 微型计算机系统 1-25
      \end{citemize}
    \item
      \textbf{1.4 微型计算机的应用 1-26}
    \end{citemize}
    
    \paragraph{PPT2(2.X)
    16位和32位微处理器}
    
    \begin{citemize}
     
    \item
      \textbf{2.1 16位微处理器8086 2-3}
    
      \begin{citemize}
       
      \item
        2.1.1 8086的编程结构 2-3
    
        \begin{citemize}
         
        \item
          总线接口部件BIU 2-5
        \item
          指令执行部件EU 2-7
        \item
          状态标志 2-10
        \item
          控制标志 2-11
        \item
          8086的总线周期的概念 2-13
    
          \begin{citemize}
           
          \item
            时钟周期、总线周期、指令周期 2-13
          \end{citemize}
        \end{citemize}
      \item
        2.1.2 8086的引脚信号和工作模式 2-16
    
        \begin{citemize}
         
        \item
          最小模式和最大模式的概念 2-16
        \item
          8086引脚信号和功能 2-17
    
          \begin{citemize}
           
          \item
            两种工作方式功能相同的引脚 2-19
          \item
            工作于最小模式下时使用的引脚 2-25
          \end{citemize}
        \item
          8086/8088 CPU在最小模式典型配置 2-28
    
          \begin{citemize}
           
          \item
            地址锁存器8282 2-30
          \item
            8286双向数据总线收发器 2-32
          \item
            时钟发生器8284A 2-35
          \end{citemize}
        \item
          8086/8088 CPU在最大模式中引脚定义 2-42
    
          \begin{citemize}
           
          \item
            最大工作模式特点 2-42
          \item
            引脚定义 2-42
          \item
            最大模式下的典型配置 2-46
          \item
            总线控制器8288 2-47
          \item
            总线控制器8288的连接 2-48
          \end{citemize}
        \end{citemize}
      \item
        2.1.3 8086的操作和时序 2-49
    
        \begin{citemize}
         
        \item
          系统的复位和启动操作 2-50
        \item
          总线操作 2-51
    
          \begin{citemize}
           
          \item
            总线读周期 2-51
          \item
            总线写周期 2-52
          \item
            总线空操作 2-53
          \end{citemize}
        \item
          8086的中断响应时序 2-54
        \item
          最小模式下的总线保持 2-55
        \item
          最大模式下的总线请求/授权 2-56
        \end{citemize}
      \item
        2.1.4 8086存储器编址和I/O编址 2-57
    
        \begin{citemize}
         
        \item
          8086的存储器编址 2-57
    
          \begin{citemize}
           
          \item
            逻辑地址和物理地址 2-58
          \item
            合成物理地址 2-60
          \end{citemize}
        \item
          8086的I/O编址 2-62
    
          \begin{citemize}
           
          \item
            和存储器统一编址 2-62
          \item
            I/O独立编址 2-64
          \end{citemize}
        \end{citemize}
      \end{citemize}
    \item
      \textbf{2.6 80x86微处理器及其发展 2-66}
    
      \begin{citemize}
       
      \item
        2.6.1 80286微处理器 2-66
      \item
        2.6.2 80386微处理器 2-69
      \item
        2.6.3 80486微处理器 2-72
      \item
        2.6.4 Pentium微处理器 2-74
      \end{citemize}
    \end{citemize}
    
    \hypertarget{ppt35.x-ux5faeux578bux8ba1ux7b97ux673aux548cux5916ux8bbeux7684ux6570ux636eux4f20ux8f93}{%
    \paragraph{PPT3(5.X)
    微型计算机和外设的数据传输}\label{ppt35.x-ux5faeux578bux8ba1ux7b97ux673aux548cux5916ux8bbeux7684ux6570ux636eux4f20ux8f93}}
    
    \begin{citemize}
     
    \item
      \textbf{5.1 为什么要用接口 3-3}
    \item
      \textbf{5.2 CPU和输入/输出设备之间的信号 3-5}
    
      \begin{citemize}
       
      \item
        5.2.1 数据信息 3-5
      \item
        5.2.2状态信息 3-5
      \item
        5.2.3 控制信息 3-5
      \end{citemize}
    \item
      \textbf{5.3 接口部件的I/O端口 3-6}
    \item
      \textbf{5.4 接口的功能以及在系统中的连接 3-8}
    
      \begin{citemize}
       
      \item
        5.4.1 接口的功能 3-8
    
        \begin{citemize}
         
        \item
          接口组成:硬件电路和软件编程 3-10
        \end{citemize}
      \item
        5.4.2 接口与系统的连接 3-11
    
        \begin{citemize}
         
        \item
          典型的I/O部件和外部电路连接图 3-11
        \end{citemize}
      \end{citemize}
    \item
      \textbf{5 .5 CPU与外设之间的数据传输方式 3-14}
    
      \begin{citemize}
       
      \item
        5.5.1程序控制方式 3-14
    
        \begin{citemize}
         
        \item
          无条件传送 3-14
        \item
          条件传送 3-17
        \item
          查询式输出的接口电路 3-19
        \item
          查询式输入的流程图 3-20
        \item
          查询式输出过程的流程图 3-21
        \end{citemize}
      \item
        5.5.2 中断方式 3-22
    
        \begin{citemize}
         
        \item
          中断传送方式的原理 3-22
        \item
          中断优先级问题的解决 3-25
        \end{citemize}
      \item
        5 .5 .3 DMA方式 3-26
    
        \begin{citemize}
         
        \item
          DMA传送方式的提出 3-26
        \item
          DMA控制器的功能和DMA传送原理 3-27
        \item
          DMA控制器的内部最小配置和接口要求 3-28
        \item
          对DMA传送对I/O接口的要求 3-30
        \item
          对DMA控制器的要求 3-30
        \item
          系统对DMA控制器和接口部件预置信息 3-31
        \item
          启动数据块输入的程序段 3-32
        \item
          DMA控制器的工作特点 3-33
        \end{citemize}
      \item
        5.5.4 输入输出过程中提出的几个问题 3-34
      \item
        5.5.5 接口和多字节数据总线的连接 3-35
      \item
        5.5.6接口部件和地址总线的错位连接 3-36
      \end{citemize}
    \end{citemize}
    
    \hypertarget{ppt47.x-ux4e2dux65adux63a7ux5236ux5668}{%
    \paragraph{PPT4(7.X)
    中断控制器}\label{ppt47.x-ux4e2dux65adux63a7ux5236ux5668}}
    
    \begin{citemize}
     
    \item
      \textbf{7-0 中断与8086中断系统 4-3}
    
      \begin{citemize}
       
      \item
        中断的基本概念 4-3
    
        \begin{citemize}
         
        \item
          中断 4-3
        \item
          中断的特点 4-4
        \item
          中断的功能 4-4
        \end{citemize}
      \item
        中断类型和中断向量表 4-6
      \item
        8086的中断分类 4-8
      \item
        中断响应过程 4-10
      \end{citemize}
    \end{citemize}
    
    \hypertarget{ppt56.x-ux4e32ux5e76ux884cux901aux4fe1ux548cux63a5ux53e3ux6280ux672f}{%
    \paragraph{PPT5(6.X)
    串并行通信和接口技术}\label{ppt56.x-ux4e32ux5e76ux884cux901aux4fe1ux548cux63a5ux53e3ux6280ux672f}}
    
    \begin{citemize}
     
    \item
      \textbf{6.1 串行接口和串行通信 5-3}
    
      \begin{citemize}
       
      \item
        6.1.1串行通信涉及的几个问题 5-3
    
        \begin{citemize}
         
        \item
          全双工、半双工和单工 5-3
        \item
          同步方式和异步方式 5-3
        \item
          串行通信的传输率 5-5
    
          \begin{citemize}
           
          \item
            波特率因子 5-6
          \item
            串行接口芯片 5-7
          \end{citemize}
        \end{citemize}
      \item
        6.1.2 串行接口 5-8
    
        \begin{citemize}
         
        \item
          可编程串行接口的典型结构 5-8
        \item
          串行接口部件4个主要寄存器 5-9
        \end{citemize}
      \end{citemize}
    \item
      \textbf{6.3 并行通信和并行接口 5-39}
    \end{citemize}
    
    \hypertarget{ppt69.x-ux8ba1ux6570ux5668ux5b9aux65f6ux5668ux63a5ux53e3ux82afux7247}{%
    \paragraph{PPT6(9.X)
    计数器/定时器接口芯片}\label{ppt69.x-ux8ba1ux6570ux5668ux5b9aux65f6ux5668ux63a5ux53e3ux82afux7247}}
    
    \begin{citemize}
     
    \item
      \textbf{9 计数器/定时器和多功能接口芯片 6-3}
    
      \begin{citemize}
       
      \item
        在微型计算机系统中经常用到定时信号 6-3
      \item
        实现定时的方法:软件和硬件两种 6-3
    
        \begin{citemize}
         
        \item
          软件定时 6-3
        \item
          硬件定时 6-4
        \end{citemize}
      \item
        9.1 可编程计数器/定时器的工作原理 6-5
    
        \begin{citemize}
         
        \item
          计数器/定时器的一些用处 6-6
        \item
          典型计数器/定时器原理图 6-7
        \end{citemize}
      \end{citemize}
    \end{citemize}
    
    \hypertarget{ppt78.x-dmaux63a7ux5236ux5668}{%
    \paragraph{PPT7(8.X) DMA控制器}\label{ppt78.x-dmaux63a7ux5236ux5668}}
    
    \begin{citemize}
     
    \item
      \textbf{8.1 DMA控制器概要 7-3}
    
      \begin{citemize}
       
      \item
        DMA控制器的传输过程 704
      \end{citemize}
    \end{citemize}
    
    \hypertarget{ppt810.x-ux6a21ux6570ux548cux6570ux8f6cux6362}{%
    \paragraph{PPT8(10.X)
    模/数和数/转换}\label{ppt810.x-ux6a21ux6570ux548cux6570ux8f6cux6362}}
    
    \begin{citemize}
    \item
      \textbf{10.1 概述 8-3}
    
      \begin{citemize}
       
      \item
        闭环实时控制系统 8-4
      \end{citemize}
    \item
      \textbf{10.2 数/模(D/A)转换器 8-5}
    
      \begin{citemize}
      \item
        10.2.1 D/A转换的原理 8-5
    
        \begin{citemize}
         
        \item
          运算放大器的工作特点和原理 8-6
        \item
          由并联电阻和运算放大器构成的 D/A转换器 8-8
        \item
          T型权电阻网络 8-9
        \end{citemize}
      \item
        10.2.2 D/A转换的指标 8-11
    
        \begin{citemize}
        \item
          分辨率、转换精度、转换速率和建立时间、
    
          线性误差、输出电平范围
        \end{citemize}
      \end{citemize}
    \item
      \textbf{10.3 模/数转换器 8-34}
    
      \begin{citemize}
       
      \item
        10.3.1 A/D转换涉及的参数 8-34
    
        \begin{citemize}
         
        \item
          分辨率、转换精度、转换率
        \end{citemize}
      \item
        10.3.2 A/D转换的方法和原理 8-35
    
        \begin{citemize}
         
        \item
          计数式A/D转换 8-35
        \item
          双积分式A/D转换(不讲)8-37
        \item
          逐次逼近式A/D转换 8-40
        \item
          用软件和D/A转换器来实现A/D转换 8-42
        \end{citemize}
      \item
        10.3.3 A/D转换器和系统连接时要考虑的问题 8-45
    
        \begin{citemize}
         
        \item
          输入模拟电压的连接 8-45
        \item
          数据输出线和系统总线的连接 8-45
        \item
          启动信号的供给 8-46
        \item
          转换结束信号以及转换数据的读取 8-46
        \item
          模拟电路和数字电路的接地问题 8-47
        \end{citemize}
      \end{citemize}
    \end{citemize}
    
    \hypertarget{ppt915.x-ux603bux7ebf-ux7406ux8bbaux4e0aux4e0dux8003}{%
    \paragraph{PPT9(15.X) 总线
    (理论上不考)}\label{ppt915.x-ux603bux7ebf-ux7406ux8bbaux4e0aux4e0dux8003}}
    
    \begin{citemize}
    \item
      总线结构优点 9-3
    \item
      15.1 总线的分类和性能指标 9-4
    
      \begin{citemize}
       
      \item
        计算机总线的分类 9-4
    
        \begin{citemize}
         
        \item
          内部总线 9-4
        \item
          局部总线 9-4
        \item
          系统总线 9-5
        \item
          外部总线 9-5
        \end{citemize}
      \item
        总线的性能指标 9-6
    
        \begin{citemize}
         
        \item
          宽度、总线频率、传输率
        \end{citemize}
      \item
        多总线技术 9-6
      \item
        总线的层次 9-7
      \item
        总线桥 9-8
      \item
        基于PC/XT总线的微机的基本结构 9-9
      \item
        基于ISA总线的微机的基本结构 9-10
      \item
        基于PCI总线的微机的基本结构 9-11
      \end{citemize}
    \item
      15.2 PCI总线的特点和系统结构 9-12
    
      \begin{citemize}
       
      \item
        15.2.1 PCI的概况和特点 9-12
    
        \begin{citemize}
         
        \item
          PCI的概况 9-12
        \item
          PCI的特点 9-12
        \end{citemize}
      \item
        15.2.2 PCI的层次化系统结构 9-14
      \end{citemize}
    \item
      15.11 PCI兼容的局部总线 9-15
    
      \begin{citemize}
      \item
        15.11.1 局部总线ISA 9-15
    
        \begin{citemize}
        \item
          16位数据线、地址线、控制线、
    
          状态线、辅助线和电源线
        \end{citemize}
      \end{citemize}
    \item
      15.12.4 通用串行总线USB 9-26
    
      \begin{citemize}
       
      \item
        USB的特点 9-27
      \item
        USB的接口特性 9-28
      \end{citemize}
    \end{citemize}
\end{multicols*}

\end{document}