\documentclass{ctexart}
\usepackage[T1]{fontenc}
\usepackage[a4paper,top=1.5cm,bottom=1.5cm,left=2cm,right=2cm,marginparwidth=1.75cm]{geometry}
\usepackage{mathtools}
\usepackage{booktabs}
\usepackage{caption}
\usepackage[colorlinks=false, allcolors=blue]{hyperref}
\renewcommand{\tableautorefname}{表}
\DeclarePairedDelimiter{\set}{\{}{\}}
\DeclarePairedDelimiter{\paren}{(}{)}

\title{微机接口第一次作业}
\author{卢雨轩 19071125}
% \date{\today}
\ctexset{
    section = {
        titleformat = \raggedright,
        name = {,},
        number = \chinese{section}、
    },
    paragraph = {
        runin = false
    },
    today = small,
    figurename = 图,
    contentsname = 目录,
    tablename = 表,
}
\makeatletter
\newcommand{\nextverbatimspread}[1]{%
  \def\verbatim@font{%
    \linespread{#1}\normalfont\ttfamily% Updated definition
    \gdef\verbatim@font{\normalfont\ttfamily}}% Revert to old definition
}
\makeatother
\begin{document}

\maketitle

\begin{enumerate}
    \item 8086系统中,设段寄存器CS=1200H,指令指针寄存器IP=FF00H,此时指令的物理地址为多少,指向此地址的段地址和偏移量是唯一的吗?

    答:物理地址为21F00H。不是唯一的。

    \item T1状态下,8086的数据/地址线上是什么信息,用哪个信号将此信息锁存起来
    
    答:总线上是要寻址的储存单元或外设的地址。用ALE信号将此信息锁存器来。
    
    \item 若当前SS=3500H,SP=0800H,说明堆栈段在存储器中的物理地址,若此时入栈10个字节,SP内容是什么?若再出栈6个字节,SP为什么值?
    
    答:物理地址为:35000H到44FFFH。入栈10字节,SP=07F6H。再出栈6字节,SP=07FCH
    
    \item 某程序数据段中存放了两个字,1EE5H和2A8CH,已知(DS)=7850H,数据存放的偏移地址为3121H及285AH。试画图说明它们在存储器中的存放情况,若要读取这两个字,需要对存储器进行几次操作?
    
    答:物理地址分别为:7B621H 7AD5AH。读取位于7B621H的字需要2次操作,读取位于7AD5AH的字需要1次操作。
    \nextverbatimspread{1}
    \begin{verbatim}
        +---------------+
  7B622 |      1E       |
        +---------------+
  7B621 |      E5       |
        +---------------+
  7B620 |      xx       |
        +---------------+
  xxxxx |      xx       |
        +---------------+
  7AD5B |      2A       |
        +---------------+
  7AD5A |      8C       |
        +---------------+
  7AD59 |      xx       |
        +---------------+
    \end{verbatim}
    
    \item 8086CPU的AD15—AD0能否直接连接到系统总线上,为什么?
    
    不可以。因为是数据、地址复用线,需要接地址、数据锁存器才能分别接入地址总线、数据总线。
\end{enumerate}

\end{document}
