% !TEX program = xelatex
\documentclass{ctexart}
\usepackage[utf8]{inputenc}
\usepackage[a4paper,top=1.5cm,bottom=1.5cm,left=2cm,right=2cm,marginparwidth=1.75cm]{geometry}
\usepackage{amsmath}
\usepackage{booktabs}
\usepackage{caption}
\usepackage[colorlinks=false, allcolors=blue]{hyperref}
\usepackage{os-common}
\renewcommand{\tableautorefname}{表}


\title{第二次操作系统作业}
\author{卢雨轩 19071125}
\ctexset{
    section = {
        titleformat = \raggedright,
        name = {,},
        number = \chinese{section}、
    },
    paragraph = {
        runin = false
    },
    today = small,
    figurename = 图,
    contentsname = 目录,
    tablename = 表,
}

\begin{document}

\maketitle

\section*{基础作业}
\begin{enumerate}
\item 论述短期、中期、长期调度之间的区别。

短期调度指从就绪队列选择进程到CPU上执行。运行频率较高,是最频繁的调度。

中期调度指换出挂起的进程,选择进入内存。
和短期调度相比,中期调度不是选择内存就绪的进程,而是将等待的进程挂起、将就绪的进程激活。
可以提高内存的利用率。

长期调度指从进程池中选择进程进入内存,搭配CPU bound与IO bound程序。提高CPU利用率与系统整体运行效率。

\item 两个进程进行上下文切换的操作
    \begin{enumerate}
        \item 『保护现场』:保存旧进程PC、所有寄存器
        \item 『切换地址空间』:换页表
        \item 『恢复现场』:恢复新进程寄存器,并跳转到新进程PC执行
    \end{enumerate}
\end{enumerate}

\section*{补充作业}

\begin{enumerate}
\item 假设有一个进程,它的工作流程是先运行150ms,然后进行I/O,最后执行250ms结束。如果系统中的进程有三个状态,当时间片为200ms时,请写出进程A从被系统接纳到运行结束所经历的状态转换并说明原因。

\begin{center}
    \begin{ganttchart}[
        canvas/.append style={fill=none, draw=black!5, line width=.75pt},
        hgrid style/.style={draw=black!5, line width=.75pt},
        vgrid={*1{draw=black!5, line width=.75pt}},
        % today=4,
        today rule/.style={
          draw=black!64,
          dash pattern=on 3.5pt off 4.5pt,
          line width=1.5pt
        },
        today label font=\small\bfseries,
        title/.style={draw=none, fill=none},
        title label font=\bfseries\footnotesize,
        title label node/.append style={below right=7pt and .75pt},
        include title in canvas=false,
        bar label font=\mdseries\small\color{black!70},
        bar label node/.append style={left=2cm},
        bar/.append style={draw=none, fill=barblue},
        bar progress label font=\mdseries\footnotesize\color{black!70},
        link/.style={-latex, line width=1.5pt, linkred},
        link label font=\scriptsize\bfseries,
        link label node/.append style={below left=-2pt and 0pt},
        x unit=1cm
      ]{1}{9}
      \gantttitle[
        title label node/.append style={below left=7pt and -3pt}
      ]{Time(ms)\ 0}{0}
      \gantttitlelist{50,100,...,450}{1} \\
      \ganttbar[
        name=PROCESS1
      ]{Process}{1}{3}
      \ganttbar[
        name=PROCESS2
      ]{}{5}{8}
      \ganttbar[
        name=PROCESS3
      ]{}{9}{9} \\
      \ganttbar[
        name=IDLE
      ]{Idle}{4}{4} \\
      \ganttbar[
        name=SCHEDULER
      ]{Scheduler}{9}{8}
      \ganttlink[
        link type=running-waiting,
        link label node/.append style=left
      ]{PROCESS1}{IDLE}
      \ganttlink[
        link type=waiting-running,
        link label node/.append style=right
      ]{IDLE}{PROCESS2}
      \ganttlink[
        link type=running-ready,
        link label node/.append style=left
      ]{PROCESS2}{SCHEDULER}
      \ganttlink[
        link type=ready-running,
        link label node/.append style=right
      ]{SCHEDULER}{PROCESS3}
    \end{ganttchart}
\end{center}
\begin{enumerate}
    \item 运行 \textrightarrow 等待:进行IO
    \item 等待 \textrightarrow 运行:IO结束
    \item 运行 \textrightarrow 就绪:时间片
    \item 就绪 \textrightarrow 运行:重新调度
\end{enumerate}
\end{enumerate}

\end{document}
