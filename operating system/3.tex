\documentclass{ctexart}
\usepackage[utf8]{inputenc}
\usepackage[a4paper,top=1.5cm,bottom=1.5cm,left=2cm,right=2cm,marginparwidth=1.75cm]{geometry}
\usepackage{amsmath}
\usepackage{booktabs}
\usepackage{caption}
\usepackage[colorlinks=false, allcolors=blue]{hyperref}
\usepackage{os-common}
\renewcommand{\tableautorefname}{表}

\title{第三次操作系统作业}
\author{卢雨轩 19071125}
\ctexset{
    section = {
        titleformat = \raggedright,
        name = {,},
        number = \chinese{section}、
    },
    paragraph = {
        runin = false
    },
    today = small,
    figurename = 图,
    contentsname = 目录,
    tablename = 表,
}

\begin{document}

\maketitle

\section*{基础作业}
\begin{outline}[enumerate]
    \1 考虑下面一组进程,进程占用的CPU区间长度以毫秒计算。假设在0时刻进程以P1, P2, P3, P4, P5的顺序到达。
\begin{center}
    \captionof{table}{进程、区间时间、优先级}
    \begin{tabular}{ccc}
        \toprule
        进程 & 区间时间 & 优先级 \\
        \midrule
        P1 & 10 & 3 \\
        P2 & 1 & 1 \\
        P3 & 2 & 3 \\
        P4 & 1 & 4 \\
        P5 & 5 & 2 \\
        \bottomrule
    \end{tabular}
\end{center}

    \2 画出4个Gantt图,分别演示使用FCFS, SJF, 非抢占优先 级(数字越小表示优先级越高)和RR(时间片=1)算法调度时 进程的执行过程。

        \3 FCFS
        \begin{figure}[H]
            \centering
            \caption{先来先服务算法甘特图}
            \begin{os-gantt}[
                y unit = .5cm,
                bar height = 1
            ]{1}{19}
                \gantttitle[
                    title label node/.append style={below left=7pt and -3pt}
                ]{Time(ms)\ 0}{0}
                \gantttitlelist{1,...,19}{1} \\
                \ganttbar{P1}{1}{10} \\
                \ganttbar{P2}{11}{11} \\
                \ganttbar{P3}{12}{13} \\
                \ganttbar{P4}{14}{14} \\
                \ganttbar{P5}{15}{19} \\
            \end{os-gantt}
        \end{figure}
        \3 SJF
        \begin{figure}[H]
            \centering
            \caption{短作业优先算法甘特图}
            \begin{os-gantt}[
                y unit = .5cm,
                bar height = 1
            ]{1}{19}
                \gantttitle[
                    title label node/.append style={below left=7pt and -3pt}
                ]{Time(ms)\ 0}{0}
                \gantttitlelist{1,...,19}{1} \\
                \ganttbar{P1}{10}{19} \\
                \ganttbar{P2}{1}{1} \\
                \ganttbar{P3}{3}{4} \\
                \ganttbar{P4}{2}{2} \\
                \ganttbar{P5}{5}{9} \\
            \end{os-gantt}
        \end{figure}
        \3 非抢占优先级调度
        \begin{figure}[H]
            \caption{非抢占优先级调度算法甘特图}
            \centering
            \begin{os-gantt}[
                y unit = .5cm,
                bar height = 1
            ]{1}{19}
                \gantttitle[
                    title label node/.append style={below left=7pt and -3pt}
                ]{Time(ms)\ 0}{0}
                \gantttitlelist{1,...,19}{1} \\
                \ganttbar{P1}{7}{16} \\
                \ganttbar{P2}{1}{1} \\
                \ganttbar{P3}{17}{18} \\
                \ganttbar{P4}{19}{19} \\
                \ganttbar{P5}{2}{6} \\
            \end{os-gantt}
        \end{figure}
        \3 RR(时间片=1)
        \begin{figure}[H]
            \caption{时间片轮转算法甘特图}
            \centering
            \begin{os-gantt}[
                y unit = .5cm,
                bar height = 1
            ]{1}{19}
                \gantttitle[
                    title label node/.append style={below left=7pt and -3pt}
                ]{Time(ms)\ 0}{0}
                \gantttitlelist{1,...,19}{1} \\
                \ganttbar{P1}{1}{1} \ganttbar{P1}{6}{6} 
                \ganttbar{P1}{9}{9} \ganttbar{P1}{11}{11} 
                \ganttbar{P1}{13}{13} \ganttbar{P1}{15}{19} \\
                \ganttbar{P2}{2}{2} \\
                \ganttbar{P3}{3}{3} \ganttbar{P1}{7}{7} 
                \ganttbar{P1}{10}{10} \ganttbar{P1}{12}{12} 
                \ganttbar{P1}{14}{14} \\
                \ganttbar{P4}{4}{4} \\
                \ganttbar{P5}{5}{5} \ganttbar{P1}{8}{8}\\
            \end{os-gantt}
        \end{figure}
    \2 每个进程的周转时间是多少?
    \2 每个进程在每种调度算法下的等待时间是多少?
    \begin{table}[H]
        \centering
        \caption{进程的周转时间与等待时间}
        \begin{tabular}{*{9}{c}}
          \toprule
          \multirow{2}{*}[-2pt]{进程} & \multicolumn{2}{c}{FCFS} & \multicolumn{2}{c}{SJF} 
            & \multicolumn{2}{c}{优先级} & \multicolumn{2}{c}{RR} \\
          \cmidrule(lr){2-3}
          \cmidrule(lr){4-5}
          \cmidrule(lr){6-7}
          \cmidrule(lr){8-9}
            & 周转时间 & 等待时间 & 周转时间 & 等待时间 & 周转时间 & 等待时间 & 周转时间 & 等待时间 \\
          \midrule
          P1 & 10ms & 0ms & 19ms & 9ms & 16ms & 6ms & 19ms & 0ms \\
          P2 & 11ms & 10ms & 1ms & 0ms & 1ms & 0ms & 2ms & 1ms \\
          P3 & 13ms & 11ms & 4ms & 2ms & 18ms & 16ms & 14ms & 2ms \\
          P4 & 14ms & 13ms & 2ms & 1ms & 19ms & 18ms & 4ms & 3ms \\
          P5 & 19ms & 14ms & 9ms & 4ms & 6ms & 1ms & 8ms & 4ms \\
          \bottomrule
        \end{tabular}
      \end{table}
\end{outline}

\section*{补充作业}


\end{document}
