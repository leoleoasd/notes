\documentclass{ctexart}
\usepackage[T1]{fontenc}
\usepackage[a4paper,top=1.5cm,bottom=1.5cm,left=2cm,right=2cm,marginparwidth=1.75cm]{geometry}
\usepackage{mathtools}
\usepackage{tikz}
\usepackage{booktabs}
\usepackage{caption}
\usepackage{outlines}
\usepackage{graphicx}
\usepackage{amsthm}
\usepackage[colorlinks=false, allcolors=blue]{hyperref}
\renewcommand{\tableautorefname}{表}
\DeclarePairedDelimiter{\set}{\{}{\}}
\DeclarePairedDelimiter{\paren}{(}{)}
\graphicspath{ {./images/} }

\title{性健康与人才发展作业 \\[.4em] \large 我佩服的一位同性人物}
\author{卢雨轩 19071125}
% \date{\today}
\ctexset{
    section = {
        titleformat = \raggedright,
        name = {,},
        number = \chinese{section}、
    },
    paragraph = {
        runin = false
    },
    today = small,
    figurename = 图,
    contentsname = 目录,
    tablename = 表,
}

\begin{document}

\maketitle
% 请写一篇文章,推荐一位你欣赏或佩服或向往或…的同性或异性人物。1人物基本情况介绍,2其有关特殊经历或事件,3你欣赏他/她…的具体内容和个人因素,4你在日常和以后生活中能做的尝试、改变行动等。内容优先、不限字数和形式,文档、图片视频、ppt等皆可。(非简答题)
我非常佩服我的高中同学,P同学。他现在就读于清华大学计算机系,参与智能产业研究院(AIR)的科研工作。

高中的时候,最初我们一起学习计算竞赛。同在竞赛班,没有基础的他每日努力学习、练习,最终进入了竞赛高级班。同时,他也对自己的情况十分了解,自知无法和初中乃至小学开始学习竞赛的同学竞争,于是早早退出竞赛,专心学习课内知识,通过高考考入清华大学计算机系。

进入大学以来,他对于自己的目标有着十分清晰的认识。大一努力学习专业课,大二上就进组开始做科研,过程中也在不断的自我督促、学习进步,到现在已经基本锁定清华智能产业研究院的直博。

我欣赏他的具体内容就是自律。P同学是一个非常自律的同学,无论是在高中学习竞赛,还是转入学习课内,抑或是在大学参与科研工作,过程中都没有人督促,自己督促自己完成科研、学习任务。而我在大一一年基本荒废,可以说什么也没学到。大二上学期加入科研项目组后直到大三才有了一点进展。我觉得,我十分需要学习他的自律。

在日后,我要尽可能的督促自己去学习、工作、科研。其实,最近我也取得了不少进步:每日坚持背单词;尽量在老师留作业的第一时间写完作业;坚持推进科研项目,与导师保持至少每周一次的沟通。在最近,除了工大校内的实验室之外,我也加入了清华大学的THUNLP实验室开展科研工作。课内学习日益紧张,课外的科研工作又逐渐变多。我也更需要督促自己高效的利用时间,完成学习、科研任务。


\end{document}
