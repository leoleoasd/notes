\documentclass{ctexart}
\usepackage[utf8]{inputenc}
\usepackage[a4paper,top=2cm,bottom=2cm,left=3cm,right=3cm,marginparwidth=1.75cm]{geometry}
\usepackage{amsmath}

\title{第一次计网作业}
\author{卢雨轩 19071125}
\date{\today}
\ctexset{
    section = {
        titleformat = \raggedright,
        name = {,},
        number = \chinese{section}、
    },
    paragraph = {
        runin = false
    },
    today = small,
    figurename = 图,
    contentsname = 目录,
    tablename = 表,
}

\begin{document}

\maketitle

\begin{itemize}
    \item [10.] 使用层次协议的两个理由是什么?
    
    降低网络设计的复杂性,可以将问题分解为更小的问题。同时,可以更换某一层的协议而不影响或修改其他层的协议。缺点是,性能可能比不分层的协议差。
    
    \item [12.] 两个网络都可以提供可靠的面向连接的服务。其中一个提供可靠的字节流,另一个提供可靠的报文流。试问这俩者是否相同?如果你认为相同,为什么要有这样的区别?如果不相同,请给出一个例子说明他们如何不同。
    
    我认为不相同。在报文流中,报文与报文之间的界线是清晰的,一次性读取一个报文不会读出两个连在一起的报文。但是在字节流中,没有清晰的报文边界,就需要应用层协议自己定义报文。
    
    『TCP协议的「粘包」问题』就是一个经典的、混淆了二者的例子。TCP协议显然是一个字节流协议,如果把他当成报文流协议,就会遇到『报文被拆开』、『两个报文一起收到』等问题。
    
    \item [15.] 在有些网络中,数据链路层处理传输错误的做法是请求发送方重传被损毁的帧。如果一帧被损坏的概率为p,试问发送一帧所需要的平均传输次数是多少?假设确认帧永远不会丢失。
    
    $$
    \begin{aligned}
    C & = 1 \times (1-p) + 2 \times p \times (1-p) + \dots \\
      & = 1 + p + p^2 + p^3 + \dots \\
      & = 1 + \sum_{i=1}^{\infty} p^i \\
      & = \frac{1}{1-p}
    \end{aligned}
    $$
    
    \item [16.] 一个系统具有$n$层协议。应用层产生长度为$M$字节的报文,在每一层加上长度为$h$字节的报文头。试问报文头所占网络带宽的比例是多少?
    
    $$
    ratio = \frac{h \times (n - 1)}{h \times (n-1) + M}
    $$
    
    \item [20.] 当两台计算机之间传输一个文件时,可以采用两种不同的确认策略。在第一种策略中,该文件被分解成许多个数据包,接受方独立的确认每一个数据包,但没有对整体文件进行确认。在第二种策略中,这些数据包没有被单独的确认,但是当整个文件到达接受方时会被确认。请讨论这两种方案。
    
    在一般的网络环境下,每个包都确认一次可以减少需要重传的次数。如果网络环境极佳(如整条链路都是自己可以控制的高品质网络),则每个文件确认一次可以减少确认包所需要的带宽。
\end{itemize}
\end{document}
