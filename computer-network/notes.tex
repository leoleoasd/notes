\documentclass{ctexart}
\usepackage[T1]{fontenc}
\usepackage[a4paper,top=1.5cm,bottom=1.5cm,left=2cm,right=2cm,marginparwidth=1.75cm]{geometry}
\usepackage{mathtools}
\usepackage{tikz}
\usepackage{booktabs}
\usepackage{caption}
\usepackage{outlines}
\usepackage{graphicx}
\usepackage{float}
\usepackage{amsthm}
\usepackage{tabu}
\usepackage{minted}
\usepackage[colorlinks=false, allcolors=blue]{hyperref}
\usepackage{cleveref}
\usepackage{underscore}
\renewcommand{\tableautorefname}{表}
\DeclarePairedDelimiter{\set}{\{}{\}}
\DeclarePairedDelimiter{\paren}{(}{)}
\graphicspath{ {./images/} }

\newcounter{fullrefcounter}
\newcommand*{\fullref}[1]{%
\addtocounter{fullrefcounter}{1}%
\label{--ref-\thefullrefcounter}%
\ifthenelse{\equal{\getpagerefnumber{--ref-\thefullrefcounter}}{\getpagerefnumber{#1}}}
  {
    \hyperref[{#1}]{\Cref*{#1} \nameref*{#1}}
  }
  {% false case
    \hyperref[{#1}]{第 \pageref*{#1} 页 \Cref*{#1} \nameref*{#1}}
  }
}

\title{计网笔记}
\author{卢雨轩 19071125}
% \date{\today}
\ctexset{
    section = {
        titleformat = \raggedright,
        name = {,},
        number = \chinese{section}、
    },
    paragraph = {
        runin = false
    },
    today = small,
    figurename = 图,
    contentsname = 目录,
    tablename = 表,
}

\begin{document}

\maketitle

\section{概述}
\subsection{计算机网络的概念和使用}
\begin{outline}
    \1 计算机网络
        \2 面向终端的计算机网络
            \3 主机 -> 终端
            \3 主机负荷重,线路利用率低。
            \3 改进:增加前端通讯处理机
        \2 计算机 -- 计算机网络
            \3 主机 -> 主机
            \3 ARPAnet
            \3 基本结构:分层。
            \4 资源子网
            \4 控制子网
            \3 里程碑:定义了网络、分类
            \3 定义了两级网络id就恶狗
            \3 报文分组交换
            \3 TCP/IP协议发展
        \2 网络体系结构研究:OSI,TCP/IP
            \3 定义各种标准
        \2 互联网应用,无线网络
    \1 计算机网络的定义
        \2 广义的观点:定义计算机通讯网络
        \2 用户透明性的观点:定义了分布式计算机网络
        \2 资源共享的观点:
            \3 目的:实现资源共享
            \3 分布在不同地理位置的多台独立计算机
        \2 与分布式系统的区别:
            \3 分布式系统的目的是提高整体性能,强调整体性
                \4 一般软件
            \3 计算机网络是共享资源
                \4 一般硬件。
    \1 计算机网络的应用
\end{outline}
\subsection{网络硬件}
\begin{outline}
    \1 网络分类 
        \2 点到点
            \3 关键:路由选择
        \2 广播
            \3 关键:介质访问控制方法
    \1 互联网络
        \2 子网和主机构成网络
        \2 子网指一组路由器和一些通信线路
        \2 多个不同的网络互相连接构成互联网
\end{outline}
\subsection{网路软件}
\begin{outline}
    \1 协议层次结构
        \2 实体:某一层的活动单元
        \2 最底层,物理层的通讯:实通信
        \2 其他层的通讯:不直接,被成为虚拟通信
        \2 相邻层之间是接口
        \2 网络体系结构:层和协议的集合
        \2 实现和接口的规范不属于网络体系结构
        \2 好处:
            \3 每一层的设计简单化
            \3 各层独立:只需要知道接口
            \3 灵活性好:每一层可以变化
        \2 协议定义两种接口:
            \3 服务接口:对上层
            \2 对等实体接口:实现对等实体之间的交换信息,本层
    \1 层次设计问题
        \2 可靠性:错误控制,路由算法
        \2 网络规模发展、互联:寻址呢命名
        \2 资源分配:多路复用,流量控制,拥塞控制,保证服务质量(实时性和带宽)
        \2 保密性,认证,完整性
    \1 面向连接和无连接
    \1 服务和协议的关系:
        \2 服务是对上一层提供的操作
        \2 协议是对同一层
        \2 协议是水平的,服务是垂直的
\end{outline}
\subsection{典型计算机网络参考模型}
\begin{outline}
    \1 OSI模型 7层
        \2 分层原则:
            \3 在不同的抽象的地方创建一层
            \3 每一层要执行明确定义的功能
            \3 每层的功能有利于指定国际标准
            \3 边界的选择要使得跨过接口的信息量尽可能少
            \3 层数不能太多
        \2 OSI参考模型
            \3 物理层
                \4 二进制位的传输
            \3 数据链路层
                \4 无差错的帧传输,差错控制,流量控制,介质访问控制
            \3 网络层
                \4 提供点到点的数据传送,选择路由和控制阻塞。
            \3 传输层:提供端到端的服务
                \4 网络层将分组传输到指定计算机
                \4 传输层将消息传送给指定程序
            \3 会话层:维护Session
            \3 表示层:抽象数据结构
            \3 应用层
    \1 TCP/IP模型:
        \2 合并会话、表示、应用
        \2 合并数据链路层和物理层
        \2 链路层:
            \3 没有描述,能传输IP分组
        \2 网络层(互联网层):
            \3 控制通信子网提供IP分组传送
            \3 使用IP地址寻址、传输数据
            \3 无连接
        \2 传输层
            \3 TCP、UDP
        \2 应用层
\end{outline}
\section{物理层}
\subsection{数据通讯的理论基础}
\begin{equation*}
    \text{最大数据速率 } = 2B \log_2 V
\end{equation*}
\begin{equation*}
    \text{最大比特率 } = B \log_2 (1 + S/N)
\end{equation*}
\subsection{有导向传输介质}
\begin{outline}
    \1 磁带、磁盘
    \1 双绞线
    \1 同轴电缆
    \1 光纤
\end{outline}
\subsection{数字调制与多路复用}

\begin{outline}
    \1 基带传输
        \2 直接编码为二进制流
    \1 通带传输
        \2 加载到交流信号上
\end{outline}
\subsection{交换}
\begin{outline}
    \1 电路交换
    \1 数据包交换
\end{outline}
\section{数据链路层}
在数据连路层上两台相邻机器之间实现可靠、有效的通信而涉及到的一些算法。 
\subsection{数据链路层的设计问题}
\begin{outline}
    \1 从网络层获取数据包,封装为帧。
    \1 提供给网络层的服务:
        \2 无确认无链接:如Ethernet
        \2 有确认无连接:如WiFi
        \2 有确认有链接
    \1 成帧:
        \2 字符计数法
            \3 帧开始表示字符数
        \2 字节填充的标志字节
            \3 两个连续FLAG表示某帧结束和下一帧的开始
            \3 包中的FLAG和ESC前面加ESC
        \2 比特填充的标志比特法
            \3 用01111110表示开始和结束
            \3 遇到连续5个1,就插入1个0
        \2 物理层的非法信号
    \1 差错控制
    \1 流量控制
\end{outline}
\subsection{协议}
\begin{outline}
    \1 乌托邦协议\ 无限制的单工协议
    \1 无错信道上的单工停-等式协议
        \2 发送每帧后等待一个确认帧
    \1 有错信道上的单工停-等式协议
        \2 引入一个计数器m,等待m的确认帧到了之后回复m+1
        \2 模2计数器就够了,m的可能取值为0或1
    \1 滑动窗口协议
        \2 1位滑动窗口协议
            \3 每帧带一个确认帧(ack)
            \3 缺点:双方同时发第一个帧,会出问题。
        \2 回退N协议
            \3 利用率小于等于$w / 1 + 2BD$。W是窗口大小,BD是带宽乘延迟除以帧大小。
            \3 窗口大小小于MAX_SEQ(也就是$2^n - 1$)
        \2 选择重传协议
            \3 选择重传:丢弃坏帧,缓存中间帧,直到重新收到好帧
            \3 否定确认:收到坏帧时回复『没收到某帧』
            \3 窗口大小小于$(MAX_SEQ + 1) / 2$ 也就是($2^{n-1}$)
\end{outline}

\section{介质访问子层}
\subsection{多路访问协议}
\subsubsection{ALOHA}
\begin{outline}
    \1 纯ALOHA
        \2 发包,失败就随机时间重传
        \2 18.4\%
    \1 分槽ALOHA
        \2 只能在每个时间槽的开始发包
        \2 36.8\%
\end{outline}
\subsubsection{载波监听多路访问控制协议}
\begin{outline}
    \1 1-坚持CSMA
        \2 如果空闲,就发送一帧。
        \2 如果忙,就等到空闲。
        \2 如果冲突,就等随机时间。
    \1 非坚持CSMA
        \2 如果忙,就等随机时间
    \1 p-坚持CSMA
        \2 如果空闲,有p的概率发送数据
        \2 有1-p的概率推迟发送
        \2 如果冲突吗重新开始
    \1 冲突检测的CSMA
        \2 如果发送过程中检测到冲突,等随机时间

\end{outline}

\subsection{无线局域网协议}
\begin{outline}
    \1 采用CSMA会导致隐藏站和暴露站问题。
    \1 MACA:避免冲突的多路访问
        \2 发送方刺激一下接受方,让接受收方发送一个短帧,这样接受方附近的站不会发帧。
        \2 如果CTS帧发送失败,重试。
\end{outline}
\subsection{数据链路层交换}
\begin{outline}
    \1 网桥:链路层交换机
    \1 学习过程:
        \2 每个网桥存一个hash table
        \2 记录每个数据包从哪儿来
        \2 需要发包的时候,找,找不到就广播
    \1 生成树过程:避免回路
        \2 广播序列号,最小的成根
        \2 每个网桥计算到根的路径,构造最短路径生成树
        \2 故障时重新计算
    \1 VLAN:染色算法
        \2 由VLAN感知交换机和主机增加、删除标记帧
\end{outline}

\section{网络层}
\end{document}
