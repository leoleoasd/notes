\documentclass{ctexart}
\usepackage[T1]{fontenc}
\usepackage[a4paper,top=1.5cm,bottom=1.5cm,left=2cm,right=2cm,marginparwidth=1.75cm]{geometry}
\usepackage{mathtools}
\usepackage{tikz}
\usepackage{booktabs}
\usepackage{caption}
\usepackage{outlines}
\usepackage{graphicx}
\usepackage{amsthm}
\usepackage[colorlinks=false, allcolors=blue]{hyperref}
\renewcommand{\tableautorefname}{表}
\DeclarePairedDelimiter{\set}{\{}{\}}
\DeclarePairedDelimiter{\paren}{(}{)}
\graphicspath{ {./images/} }

\title{计算机网络第三次作业}
\author{卢雨轩 19071125}
% \date{\today}
\ctexset{
    section = {
        titleformat = \raggedright,
        name = {,},
        number = \chinese{section}、
    },
    paragraph = {
        runin = false
    },
    today = small,
    figurename = 图,
    contentsname = 目录,
    tablename = 表,
}

\begin{document}

\maketitle

\begin{outline}[enumerate]
    \1[2.] N个站共享一个56kbps的纯ALOHA信道。每个站平均每100秒发出一个1000位
    长的帧,即使前面的帧还没有被发送出去。试问N最大值是多少?

    可利用的信道$56kbps \times 0.184 = 10.3kbps$。每个站的平均传输速率
    为$1000bps / 100 = 10bps$。因此$N = 10.3kbps / 10bps = 1030$
    \1[17.] 一个通过以太网发送的IP数据包长16字节,其中包括所有的头。如果没有使用LLC,试问需要填补字节吗?
    如果需要,需要填补多少个字节?

    长度大于46字节,不需要
    \1[18.] 以太网帧必须至少64字节长,才能确保当电缆另一端发生冲突时,发送方
    仍处于发送过程中。快速以太网也有同样的64字节最小帧长度限制,但是他可以以
    快10被的速度发送数据。试问它如何有可能维持同样地最小帧长度限制?

    最长电缆长度缩小10倍

    \1[38.] 用图4-41(b)用网桥B1和B2链接的拓展局域网。假设两个网桥的哈希表
    是空的,对于下面的传输序列,请列出转发数据包所用端口。
        \2 A发送一个数据包给C。

        B1 转发给 2,3,4。B2转发给1,2,3。

        \2 E发送一个数据包给F。

        B2 转发给 1,3,4。B1转发给1,2,3。

        \2 F发送一个数据包给E。

        B2 不转发。

        \2 G发送一个数据包给E。

        B2转发给2。

        \2 D发送一个数据包给A。

        B2转发到4,B1转发到1。

        \2 B发送一个数据包给F。

        B1转发到1,3,4,B2转发到2。
    \1[41.] 为了使得VLAN正常工作,在网桥内部需要有相应的配置表。如果图4-47
    中的VLAN使用集线器而不是交换机,情况会怎么样呢?集线器也需要配置表吗?

    不能正常工作,不需要配置。

    \1[42.] 在图4-48中,右侧传统终端域中的交换机是一个VLAN干支交换机。请问有
    可能在那里使用普通的交换机吗?如果可能,试问他如何工作?

    能。所有和他相连的设备都处于一个VLAN中,所有到达的和离开的帧都没有VLAN 标记。
\end{outline}

\end{document}
