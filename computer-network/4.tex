\documentclass{ctexart}
\usepackage[T1]{fontenc}
\usepackage[a4paper,top=1.5cm,bottom=1.5cm,left=2cm,right=2cm,marginparwidth=1.75cm]{geometry}
\usepackage{mathtools}
\usepackage{tikz}
\usepackage{booktabs}
\usepackage{caption}
\usepackage{outlines}
\usepackage{graphicx}
\usepackage{amsthm}
\usepackage[colorlinks=false, allcolors=blue]{hyperref}
\renewcommand{\tableautorefname}{表}
\DeclarePairedDelimiter{\set}{\{}{\}}
\DeclarePairedDelimiter{\paren}{(}{)}
\graphicspath{ {./images/} }

\title{计算机网络第四次作业}
\author{卢雨轩 19071125}
% \date{\today}
\ctexset{
    section = {
        titleformat = \raggedright,
        name = {,},
        number = \chinese{section}、
    },
    paragraph = {
        runin = false
    },
    today = small,
    figurename = 图,
    contentsname = 目录,
    tablename = 表,
}

\begin{document}

\maketitle

\begin{outline}[enumerate]
    \1[6.]
    B:(11,6,14,18,12,8)\\
    D:(19,15,9,3,9,10) \\
    E:(12,11,8,14,5,9) \\
    总:(11,6,0,3,5,8) \\
    路由表:(B,B,-,D,E,B)
    \1[19.] 
    8MB / (6mbps - 1mbps) = 12.8s.
    \1[27.]
    194.47.21.130
    \1[28.]
    $2^12 = 4096$
    \1[30.]
    A: \texttt{198.16.0.0/20} \\
    B: \texttt{198.16.16.0/21} \\
    C: \texttt{198.16.32.0/20} \\
    D: \texttt{198.16.64.0/19} \\
    \1[31.]
    可以,\texttt{57.6.96.0/19}
    \1[32.]
    没有。如果新分配的地址块的数据包到来,会优先匹配后缀更长的路由。
    \1[33.]
    \2[(a)] Interface 1
    \2[(b)] Interface 0
    \2[(c)] Router 2
    \2[(d)] Router 1
    \2[(e)] Router 2
\end{outline}

\end{document}
