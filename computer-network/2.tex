\documentclass{ctexart}
\usepackage[T1]{fontenc}
\usepackage[a4paper,top=1.5cm,bottom=1.5cm,left=2cm,right=2cm,marginparwidth=1.75cm]{geometry}
\usepackage{mathtools}
\usepackage{booktabs}
\usepackage{caption}
\usepackage[colorlinks=false, allcolors=blue]{hyperref}

\DeclarePairedDelimiter{\set}{\{}{\}}
\DeclarePairedDelimiter{\paren}{(}{)}

\renewcommand{\tableautorefname}{表}

\title{第二次计算机网络作业}
\author{卢雨轩 19071125}
% \date{\today}
\ctexset{
    section = {
        titleformat = \raggedright,
        name = {,},
        number = \chinese{section}、
    },
    paragraph = {
        runin = false
    },
    today = small,
    figurename = 图,
    contentsname = 目录,
    tablename = 表,
}

\begin{document}

\maketitle

\begin{itemize}
    \item[1.] 一个上层数据包被分成了10个帧,每一帧有80\%的机会无损地到达目的地。
    如果数据链路层协议没有提供错误控制,试问,该报文平均需要发送多少次才能完整的到达接受方?

    整个数据包传输失败的概率p为:
    \begin{equation*}
        p = 1 - 0.8^{10} \approx 0.893
    \end{equation*}
    所以整个报文发送的次数为:
    \begin{align*}
        T & = \sum_{i=1}^{\infty}i\paren{1-p}p^{i-1} \\
          & = (1-p)\sum_{i=1}^{\infty}ip^{i-1} \\
          & = 1-p + 2p(1-p) + 3p^2(1-p) + \dots \\
          & = 1 + p + p^2 + p^3 + \dots \\
          & = \frac{1}{1-p} \\
          & \approx 9.313
    \end{align*}
    \item[2.] 数据链路协议使用了下面的字符编码:
    \begin{center}
        A: 01000111;\quad B:11100011; 
        \quad FLAG: 01111110;\quad ESC: 11100000
    \end{center} 
    为了传输一个包括4个字符的帧: A B ESC FLAG,试问使用下面的成帧方法时所发送
    的比特序列是什么?
    \begin{enumerate}
        \item 字节计数
        \begin{equation*}
            \underbrace{0000 0101}_{5}
            \underbrace{01000111}_A
            \underbrace{11100011}_B
            \underbrace{11100000}_{ESC}
            \underbrace{01111110}_{FLAG}
        \end{equation*}
        \item 字节填充的标志字节
        \begin{equation*}
            \underbrace{01111110}_{FLAG}
            \underbrace{01000111}_A
            \underbrace{11100011}_B
            \underbrace{11100000}_{ESC}
            \underbrace{11100000}_{ESC}
            \underbrace{11100000}_{ESC}
            \underbrace{01111110}_{FLAG}
            \underbrace{01111110}_{FLAG}
        \end{equation*}
        \item 比特填充的头尾标志字节
        \begin{equation*}
            \underbrace{01111110}_{BEGIN}
            \underbrace{01000111}_A
            \underbrace{11\underline{0}100011}_B
            \underbrace{111\underline{0}00000}_{ESC}
            \underbrace{011111\underline{0}10}_{FLAG}
            \underbrace{01111110}_{END}
        \end{equation*}
        (插入的$0$用$\underline{0}$表示)
    \end{enumerate}
    \item[3.] 一个数据流中出现了这样的数据段: A B ESC C ESC FLAG FLAG D,
    假设采用本章介绍的字节填充算法,试问经过填充之后输出的是什么?
    
    A B ESC ESC C ESC ESC ESC FLAG ESC FLAG D

    \item[6.] 需要在数据链路层上被发送一个比特串:
    011110111110111110,试问,经过比特填充后实际发送出去的是什么?

    $01111011111\underline{0}011111\underline{0}0$

    \item[20.] 考虑一个具有4kbps速率和20毫秒传输延迟的信到。试问帧的大小
    在什么范围内,停-等式协议才能获得至少50\%的效率?

    设以4kbps速率发送该帧所需要的时间为$x$。则有:
    \begin{equation*}
        \frac{x}{x + 40ms} \geq 0.5
    \end{equation*}
    所以$x \geq 40ms$,即帧的大小$size \geq 164 bits$

    \item[27.] 地球到一个遥远的行星的距离大约是$9\times10^{10}$米。如果采用
    停-等试协议在一条64mbps的点到颠连路上传输帧,请问信到的利用率是多少?假设帧
    大小为32KB,光速为$3 \times 10^8 m/s$

    发送帧所需要的时间$T = 32KB / 64mbps + \frac{9\times10^{10}}{3\times10^8}
    = 0.0039 s + 300s = 300.0039s.
    $

    回收确认帧所需要的时间$T_2 = \frac{9\times10^{10}}{3\times10^8} = 300s$

    利用率$r = \frac{0.0039}{600.0039} = 6.5103 \times 10^{-6}$

    \item[34.] 考虑在一个无错的64kbps卫星信到上单项发送512字节长的数据帧,来自另一个方向
    反馈的确认帧非常短。对于窗口大小为1、7、15、127的情形,试问最大地吞吐量分别是多少?
    从地球到卫星的传播时间为270毫秒。

    发送单个帧的时间$t = 512B / 64kbps = 0.0625s = 62.5 ms$

    完成传输单个帧的时间$T = 62.5 + 2 \times 270 = 602.5 ms$

    窗口大小为1时,传输读率为$r = 512B / T \approx 6.79 kbps$

    窗口大小为7时,传输速率为$r = 512B \times 7 / T \approx 47.58 kbps$

    窗口大小为15时,传输速率为$r = 512B \times 15 / T > 64kbps$,大于信道容量,因此为$64kbps$。
    
    同理,窗口大小为127时,传输速率也是$64kbps$.
\end{itemize}

\end{document}
