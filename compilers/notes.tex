\documentclass[oneside]{ctexbook}
\usepackage[T1]{fontenc}
\usepackage{ctex}
\usepackage[a4paper,top=1.5cm,bottom=1.5cm,left=2cm,right=2cm,marginparwidth=1.75cm]{geometry}
\usepackage{mathtools}
\usepackage{tikz}
\usepackage{booktabs}
\usepackage{caption}
\usepackage{outlines}
\usepackage{graphicx}
\usepackage{float}
\usepackage{amsthm}
\usepackage{tabularray}
\usepackage{minted}
\usepackage[colorlinks=false, allcolors=blue]{hyperref}
\usepackage{cleveref}
\usepackage{wrapfig}

\xeCJKsetup{CJKmath=true}

\usetikzlibrary{fit}
\usetikzlibrary{calc}
\usetikzlibrary{arrows}
\usetikzlibrary{positioning}
\renewcommand{\tableautorefname}{表}
\DeclarePairedDelimiter{\set}{\{}{\}}
\DeclarePairedDelimiter{\paren}{(}{)}
\graphicspath{ {./images/} }

\newtheorem{definition}{定义}[section]
\newtheorem{example}{例}[section]
\newtheorem{theorem}{定理}[section]
\newtheorem{lemma}{引理}[section]
\crefname{equation}{方程}{方程}
\crefname{algorithm}{算法}{算法}
\crefname{lemma}{引理}{引理}
\crefname{table}{表}{表}
\crefname{figure}{图}{图}
\crefname{example}{例}{例}
\newcounter{fullrefcounter}
\newcommand*{\fullref}[1]{%
\addtocounter{fullrefcounter}{1}%
\label{--ref-\thefullrefcounter}%
\ifthenelse{\equal{\getpagerefnumber{--ref-\thefullrefcounter}}{\getpagerefnumber{#1}}}
  {
    \hyperref[{#1}]{\Cref*{#1} \nameref*{#1}}
  }
  {% false case
    \hyperref[{#1}]{第 \pageref*{#1} 页 \Cref*{#1} \nameref*{#1}}
  }
}
\tikzstyle{arrow} = [-latex',->, to path={-- (\tikztotarget)}]
\tikzstyle{darrow} = [-latex',->, to path={-- (\tikztotarget)},double]

\newcommand{\tdiagram}[5][]{
\begin{scope}[local bounding box={#1},shift={#2}]
    \draw [thick] (-2, 1) --  (2, 1) -- node[left] {#5} (2, 0) --  (1, 0) -- (1, -1) -- node[above]{#4} (-1, -1) -- (-1, 0) -- (-2, 0) -- node[right]{#3} cycle;
\end{scope}
}


\title{编译原理课程笔记}
\author{卢雨轩 19071125}
% \date{\today}
\ctexset{
    paragraph = {
        runin = false
    },
    today = small,
    figurename = 图,
    contentsname = 目录,
    tablename = 表,
}

\begin{document}

\maketitle

\chapter{绪论}

\section*{主要内容}
\begin{tblr}{
  column{1}={preto=\textbullet}
}
    编译原理及其设计概述  & 3学时        \\
    语言与文法 &5学时     \\
    词法问题   & 6学时   \\
    语法分析  & 13学时 &自顶向下(LL);自底向上(LR); \\
    语义分析 & 10学时 &\\
    运行环境 &3学时 & 过程调用,符号管理 \\
    优化与代码生成 & 2学时 \\
    总结 \\
\end{tblr}

\section{计算机语言的发展}
\begin{outline}
    \1 机器语言\quad Machine Language
        \2 二进制代码 \quad Binary Code
    \1 汇编语言\quad Assemble Language
        \2 二进制代码与助记符
        \2 接近计算机硬件指令系统
    \1 高级语言\quad High Level Language
        \2 语句定义数据、描述算法
    \1 命令语言\quad Command
        \2 功能封装
    \1 高级语言的分类
        \2 命令式语言 Imperative Language
            \3 Fortran, Basic, Pascal, C, COBOL, ALGOL
        \2 函数式语言 Functional Language
            \3 LISP, ML
        \2 逻辑式语言 Logical Language
            \3 Prolog
        \2 面向对象语言
\end{outline}
\section{翻译系统}
\begin{outline}
    \1 翻译程序 -- Translator
        \2 将某种语言描述的程序(源程序,Source Code)翻译为\emph{等价}的另一种语言描述的程序(目标程序,Object Code)的程序。
    \1 编译程序 -- Compiler
        \2 将某一种高级语言描述的程序翻译为汇编、机器语言描述的程序
    \1 解释程序 -- Interpreter
    \1 编译系统 = 编译程序 + 运行系统
    \1 其他翻译程序
        \2 诊断编译程序
        \2 优化编译程序
        \2 交叉编译程序
        \2 可变目标编译程序
        \2 并行编译程序
        \2 汇编程序、交叉汇编程序、反汇编程序
\end{outline}
\begin{figure}[htp]
    \centering
    \begin{tikzpicture}[node distance=1.5cm]
        \node (sp) {Source Program};
        % \node (compiler) [right=1 cm of sp] {Compiler};
        \node (OP) [below of=sp] {Object Program};
        \node (RS) [below of=OP] {Run System};
        \node [right=0 cm of RS]{(支撑环境,运行时库)};
        \node (input) [left=.4cm of RS] {Input};
        \node (output) [below of=RS] {Output};
        \draw [->,double] (sp) -- node[right] {Compiler} (OP);
        \draw [->,double] (OP) -- (RS);
        \draw [->] (input) -- (RS);
        \draw [->,double] (RS) -- (output);
    \end{tikzpicture}
    \caption{编译系统 = 编译程序 + 运行系统}
\end{figure}
\section{翻译系统的功能分析}
\begin{outline}
    \1 分析
        \2 词法,语法,语义
    \1 翻译
        \2 语句的翻译,代码生成
    \1 例如:标识符左值与右值的绑定
        \1 变量:存储单元\;\;\qquad 名字:值
        \1 函数:目标代码序列\; 名字:入口地址
\end{outline}
\section{翻译系统的总体结构}
\subsection{词法分析}

\begin{wrapfigure}[8]{r}{.5\linewidth}
    \centering
    \begin{tikzpicture}
        \node (0) {};
        \node (lexer) [below=.5cm of 0]{词法分析器 Lexical Analyzer};
        \node (parser) [below=.5cm of lexer] {语法分析器 Syntax Analyzer};
        \node (semantic) [below=.5cm of parser] {语义分析 Semantic Analyzer};
        \node (intermediate) [below = .5cm of semantic] {中间代码生成 Intermediate Code Generate};
        \node (optimise) [below= .5cm of intermediate] {代码优化器 Optimiser};
        \node (1) [below=.5cm of optimise] {目标代码生成器 Object Code Generator};
        \node (2) [below=.5cm of 1] {};
        \draw [darrow] (0) -- node[right]{源程序} (lexer);
        \draw [darrow] (lexer) -> node[right]{单词符号} (parser);
        \draw [darrow] (parser) -> node[right]{语法单元(AST)} (semantic) ;
        \draw [darrow] (semantic) -> node[right]{语法单元(AST)} (intermediate) ;
        \draw [darrow] (intermediate) -> node[right]{中间代码} (optimise) ;
        \draw [darrow] (optimise) -> node[right]{中间代码} (1) ;
        \draw [darrow] (1) -> node[right]{目标代码} (2) ;   

        
        \node [draw=black, thick, densely dotted,inner ysep=0,fit=(0) (lexer) (parser) (semantic) (intermediate) (optimise),label=left:{前端},color=purple!70!white] {};

        \node [draw=red!30!black, thick, densely dotted,inner ysep=0,fit=(optimise) (1) (2),label=left:{后端}] {};
    \end{tikzpicture}
\end{wrapfigure}
词法分析器(Lexical Analyzer)完成词法分析。扫描源程序,并转换为Token串,同时查找语法错误,进行标识符登记(管理符号表)。

输入: 字符串。

输出:(种类码,属性值)对。

\subsection{语法分析}

语法分析器(Syntax Analyzer)完成语法分析。实现『组词成句』,构造分析树,指出语法错误并指导翻译。

输入:Token序列

输出:语法成分(抽象语法树)

\subsection{语义分析}

语义分析器(Semantic Analyzer)分析由语法分析器给出的语法单位的语义。
\begin{outline}
    \1 获取标识符的属性:类型、作用域等
    \1 语义检查:运算合法、取值范围
    \1 子程序的静态绑定:代码的相对地址
    \1 变量的静态绑定:数据的相对地址
\end{outline}

\subsection{中间代码生成}
升层中间代码,如:前、后缀表达式、三地址表示、LLVM IR。

\subsection{代码优化}
对中间代码的优化处理:提高运行速度、节省储存空间
\begin{outline}
    \1 与机器无关的优化
        \2 常量合并
        \2 公共子表达式提取等
    \1 与机器有关的优化
        \2 循环展开
        \2 向量化
        \2 访存优化
        \2 寄存器排布
\end{outline}

\subsection{目标代码生成}
将中间代码转换为目标机器上的指令代码或者汇编代码。

\subsection{表格管理}
管理编译过程中的各种符号表,辅助完成语法、语义检查,完成静态绑定,管理编译过程。

\subsection{错误处理}
进行错误的检查、报告以及纠正。

\begin{outline}
    \1 词法错误:拼写、定义;
    \1 语法错误:语句结构、表达式结构;
    \1 语义错误:表达式类型不匹配;
\end{outline}

\subsection{前端与后端}
将编译过程分为前端与后端。其中:
\begin{outline}
    \1 前端:与源语言有关,与目标机器无关的部分
        \2 词法分析、语法分析、语义分析、中间代码生成、与机器无关的优化
    \1 后端:与目标机器有关的部分
        \2 与机器有关的代码优化、目标代码生成
\end{outline}
好处:实现新语言和新机器,只需要实现前端、后端中的某一个。


\section{编译程序的生成}
通过自动化技术来生成编译程序。

\subsection{T形图}

\begin{wrapfigure}[3]{r}{0.3\linewidth}
    \centering
    \vspace{-2\baselineskip}
    \begin{tikzpicture}
        \tdiagram{(0,0)}{源语言}{表示语言}{目标语言}
    \end{tikzpicture}
\end{wrapfigure}
用T形图表示语言翻译:从源语言翻译为目标语言,而翻译器本身是表示语言。

\subsection{交叉编译(Cross Compiling)、移植}
\begin{example} \label{example1}
    A机上有一个C语言编译器,是否可以利用此编译器实现B机器上的C语言编译器?
\end{example}
\begin{figure}[t]
    \centering
    \begin{tikzpicture}
        \tdiagram[p0]{(-3, 0)}{C语言}{C语言}{B机器}
        \node [below=.5cm of p0] {P0\textsubscript{(编写)}};
        \tdiagram[p1]{(0, -1)}{C语言}{A机器}{A机器}
        \node [below=.5cm of p1] {P1\textsubscript{(原有)}};
        \tdiagram[p2]{(3, 0)}{C语言}{A机器}{B机器}
        \node [below=.5cm of p2] {P2\textsubscript{(生成)}};
    \end{tikzpicture}
    \begin{tikzpicture}
        \tdiagram[p0]{(-3, 0)}{C语言}{C语言}{B机器}
        \node [below=.5cm of p0] {P0\textsubscript{(编写)}};
        \tdiagram[p2]{(0, -1)}{C语言}{A机器}{B机器}
        \node [below=.5cm of p2] {P2\textsubscript{(生成)}};
        \tdiagram[p3]{(3, 0)}{C语言}{B机器}{B机器}
        \node [below=.5cm of p3] {P3\textsubscript{(生成)}};
    \end{tikzpicture}
    \caption{\protect\fullref{example1} 图例}
\end{figure}
\begin{outline}[enumerate]
    \1  用C语言编制B机器的编译程序P0(C $\rightarrow$ B)
    \1  用P1编译P0,得到P2
    \1  用P2编译P0,得到P3
\end{outline}

\chapter{高级语言及其文法}
\section{语言概述}
\begin{outline}
    \1 什么是语言
        \2 自然语言(Natural Language)
            \3 是\emph{人与人}之间的通讯工具
            \3 语义(Semantics):环境,背景知识,语气,二义性,难以形式化
        \2 计算机语言(Computer Language)
            \3 计算机系统之间、人机之间的通讯工具
            \3 严格的语法(Grammar)、语义(Semantics)
                \4 易于形式化,从而实现自动化。
    \1 语言的描述方法
        \2 人类『已有』的语言基础
            \3 自然语言:自然、方便 —— 非形式化
            \3 数学语言(符号):严格,准确 —— 形式化
        \2 机器要『掌握』规则:形式化描述
            \3 高度抽象 —— 表示(representation)
            \3 方便计算机表示 —— 有穷描述(finite description)
            \3 扎实的理论基础 —— 结构(structure)
    \1 语言 —— 形式化内容的提取
        \2 字符(Character):语言中\emph{允许}出现的基本字符
        \2 单词(Token):满足一定\emph{规则}的字符串
        \2 句子(Sentence):满足一定规则的单词序列
        \2 语言(Language):满足一定规则的句子集合
    \1 \emph{语言是字和组合字的规则 —— 结构性描述}
    \1 程序设计语言 —— 形式化内容的提取
        \2 字符(Character):语言中可出现的所有字符——基本字符集
        \2 单词(Token):满足\emph{词法规则}的字符串(RL)
        \2 语句(Sentence):满足\emph{语法规则}的单词序列(CFL)
        \2 程序(Program):满足\emph{语法规则}的单词序列(CFL)
        \2 程序设计语言:组成程序的所有语句的集合(CFL)
    \1 描述形式 —— 文法
        \2 语法 —— 语句
            \3 语句的组成规则
            \3 描述方法:BNF范式(Backus Normal Form,Backus-Naur Form)、语法描述图
\end{outline}
\section{基本定义}
\begin{definition}
    字母表(Alphabet) $\Sigma$ 是一个非空有穷集合,字母表中的元素称为字母表的一个字母(Letter),也叫字符(Character)
\end{definition}
\begin{definition}
    字母表$\Sigma$上的符号串(String)
    \begin{outline}[enumerate]
        \1 $\epsilon$ 是 $\Sigma$ 上的一个符号串(叫做空串);
        \1 若$x$是$\Sigma$上的字符串,而$a \in \Sigma$,则$xa$是$\Sigma$上的符号串;
        \1 $y$是$\Sigma$上的符号串,当且仅当它由 (1) 和 (2) 导出
    \end{outline}

    非形式化描述:由字母表中的符号组成的任何有穷序列被成为在该字母表上的符号串,也称作『字』(Word)
\end{definition}

\begin{definition}
    设 $\Sigma_1,\Sigma_2$ 是两个字母表,$\Sigma_1$ 与 $\Sigma_2$ 的\emph{乘积(Product)} $\Sigma_1 \Sigma_2 = \set{ab | a \in \Sigma_1, b \in \Sigma_2}$
\end{definition}

\begin{definition}
    $\Sigma$的幂(Power):
    \begin{outline}[enumerate]
        \1 $\Sigma^0 = \set{\epsilon}$
        \1 $\Sigma^n = \Sigma^{n-1}\Sigma$ 
    \end{outline}
\end{definition}

\begin{definition}
    $\Sigma$的\emph{正闭包(Positive Closure)}:
    \begin{outline}[enumerate]
        \1 $\Sigma^+ = \set{x|x 是 \Sigma 上的非空字符串}$
        \1 $\Sigma^+ = \Sigma \cup \Sigma^2 \cup \Sigma^3 \cup \dots$
    \end{outline}

    $\Sigma$的\emph{克林闭包(Kleene Closure)}:
    
    $\begin{aligned}
        \Sigma^* &= \set{\epsilon} \cup \Sigma^+ \\
                 &= \Sigma^0 \cup \Sigma \cup \Sigma^2 \cup \Sigma^3 \cup \dots
    \end{aligned}$
\end{definition}

\begin{definition}
    设 $\Sigma$ 是一个字母表, $\forall L \subseteq \Sigma^*$,$L$称为字母表$\Sigma$的一个\emph{语言(Language)},$\forall x \in L$,$x$叫做$L$的一个\emph{句子}。
\end{definition}

\begin{definition}
    设$s$是字符串:
    \begin{outline}
        \1 \emph{前缀}:移去$s$的尾部的若干(含0)个字符
        \1 \emph{后缀}:移去$s$的头部的若干(含0)个字符
        \1 \emph{子串}:从$s$中删去一个前缀和一个后缀
        \1 \emph{字序列}:从$s$中删去0个或多个符号,\emph{不要求连续}
        \1 \emph{长度}:符号串中符号的数目
    \end{outline}
\end{definition}
\begin{definition}
    设$x$和$y$是符号串,他们的\emph{连接(Concatenation)}$xy$是把$y$的符号写在$x$的符号之后得到的符号串
\end{definition}
\begin{definition}
    设$x$是符号串,$x$的$n$次\emph{幂(Power)}为:
    \begin{outline}
        \1 $x^0 = \epsilon$
        \1 $x^n = x^{n-1} x$
    \end{outline}
\end{definition}
\section{文法的定义}
\emph{如何实现语言结构的形式化描述?}


\end{document}
