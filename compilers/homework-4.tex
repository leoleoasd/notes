\documentclass{ctexart}
\usepackage[T1]{fontenc}
\usepackage[a4paper,top=1.5cm,bottom=1.5cm,left=2cm,right=2cm,marginparwidth=1.75cm]{geometry}
\usepackage{mathtools}
\usepackage{tikz}
\usepackage{booktabs}
\usepackage{caption}
\usepackage{outlines}
\usepackage{graphicx}
\usepackage{float}
\usepackage{amsthm}
\usepackage{tabularray}
\usepackage{minted}
\usepackage[colorlinks=false, allcolors=blue]{hyperref}
\usepackage{cleveref}
\usepackage{amssymb}
\usepackage{algorithm}
\usepackage{algorithmicx}
\usepackage{algpseudocode}
\renewcommand{\tableautorefname}{表}
\DeclareMathOperator{\firstop}{FIRSTOP}
\DeclareMathOperator{\lastop}{LASTOP}
\UseTblrLibrary{booktabs}
\DeclarePairedDelimiter{\set}{\{}{\}}
\DeclarePairedDelimiter{\paren}{(}{)}
\graphicspath{ {./images/} }

\newcounter{fullrefcounter}
\newcommand*{\fullref}[1]{%
\addtocounter{fullrefcounter}{1}%
\label{--ref-\thefullrefcounter}%
\ifthenelse{\equal{\getpagerefnumber{--ref-\thefullrefcounter}}{\getpagerefnumber{#1}}}
  {
    \hyperref[{#1}]{\Cref*{#1} \nameref*{#1}}
  }
  {% false case
    \hyperref[{#1}]{第 \pageref*{#1} 页 \Cref*{#1} \nameref*{#1}}
  }
}

\title{编译原理第 4 次作业}
\author{卢雨轩 19071125}
% \date{\today}
\ctexset{
    section = {
        titleformat = \raggedright,
        name = {,},
        number = \chinese{section}、
    },
    paragraph = {
        runin = false
    },
    today = small,
    figurename = 图,
    contentsname = 目录,
    tablename = 表,
}

\begin{document}

\maketitle

\begin{outline}
    \1[6.] 求LASTOP \par

\begin{algorithm}[H]
    \caption{求Lastop}
    \begin{algorithmic}[1]
        \label{algo:1}
        \Procedure{LASTOP}{}
            \For{$A \to \dots a \in P$}
                \State $\lastop(A) \gets \lastop(A) \cup \set{a}$
            \EndFor
            \For{$A \to \dots aB \in P$}
                \State $\lastop(A) \gets \lastop(A) \cup \set{a}$
            \EndFor
            \For{$A \to \dots B \in P$}
                \State $\lastop(A) \gets \lastop(A) \cup \lastop(B)$
            \EndFor
        \EndProcedure
    \end{algorithmic}
\end{algorithm}
    \1[9.] 文法$G = \begin{cases}
        S \to S; G | G \\
        G \to G(T) | H \\
        H \to a | (S) \\
        T \to T + S | S \\
    \end{cases}$
        \2[(1)] 构造G的算符优先关系表,判断G是否为算符优先文法
        
        \begin{table}[H]
            \centering
            \begin{tblr}{
                colspec={X[c]X[c]X[c]},
                width=.5\linewidth,
                cell{2-5}{2-3}={mode=math}
            }
                \toprule
                  & FIRSTOP{} & LASTOP \\
                \midrule
                S & \set{a, (, ;} & \set{a, ), ;} \\
                G & \set{a, (}    & \set{a, )} \\
                H & \set{a, (}    & \set{a, )} \\
                T & \set{a, (, ;, +} & \set{a, ), +, ;} \\
                \bottomrule
            \end{tblr}
        \end{table}
        \begin{table}[H]
            \centering
            \begin{tblr}{
                hlines,vlines,
                cells={mode=math},
                colspec={X[c]X[c]X[c]X[c]X[c]X[c]X[c]},
                width=.6\linewidth
            }
                  & a & ( & ) & ; & + & \# \\
                a &   & \gtrdot & \gtrdot & \gtrdot & \gtrdot & \gtrdot \\
                ( & \lessdot & \lessdot & \equiv & \lessdot & \lessdot  \\
                ) &  & \gtrdot & \gtrdot & \gtrdot & \gtrdot& \gtrdot \\
                ; & \lessdot & \lessdot & \gtrdot & \gtrdot & \gtrdot & \gtrdot \\
                + & \lessdot & \lessdot & \gtrdot & \lessdot & \gtrdot \\
                \# & \lessdot & \lessdot &  & \lessdot & &acc \\
            \end{tblr}
        \end{table}
        两个符号之间至多只有一种关系,是算符优先文法。
        \2[(2)] 给出句型$a(T+S);H;(S)$的短语、句柄、素短语和最左素短语。

            短语: $\set{a, T+S, a(T+S), a(T+S);H, H, (S), a(T+S);H;(S)}$

            句柄: $\set{a}$

            素短语: $\set{a, (T+S), (S)}$

            最左素短语: $\set{a}$
\end{outline}

\end{document}
