\documentclass{ctexart}
\usepackage[T1]{fontenc}
\usepackage[a4paper,top=1.5cm,bottom=1.5cm,left=2cm,right=2cm,marginparwidth=1.75cm]{geometry}
\usepackage{mathtools}
\usepackage{tikz}
\usepackage{booktabs}
\usepackage{caption}
\usepackage{outlines}
\usepackage{graphicx}
\usepackage{float}
\usepackage{amsthm}
\usepackage{tabularray}
\usepackage{minted}
\usepackage[colorlinks=false, allcolors=blue]{hyperref}
\usepackage{cleveref}
\usepackage{gbt7714}
\bibliographystyle{gbt7714-numerical}
\renewcommand{\tableautorefname}{表}
\DeclarePairedDelimiter{\set}{\{}{\}}
\DeclarePairedDelimiter{\paren}{(}{)}
\graphicspath{ {./images/} }

\newcounter{fullrefcounter}
\newcommand*{\fullref}[1]{%
\addtocounter{fullrefcounter}{1}%
\label{--ref-\thefullrefcounter}%
\ifthenelse{\equal{\getpagerefnumber{--ref-\thefullrefcounter}}{\getpagerefnumber{#1}}}
  {
    \hyperref[{#1}]{\Cref*{#1} \nameref*{#1}}
  }
  {% false case
    \hyperref[{#1}]{第 \pageref*{#1} 页 \Cref*{#1} \nameref*{#1}}
  }
}

\title{我国基础软件现状,以编译器为例 \\
\large 编译原理第一次作业}
\author{卢雨轩 19071125}
% \date{\today}
\ctexset{
    section = {
        titleformat = \raggedright,
        name = {,},
        number = \chinese{section}、
    },
    paragraph = {
        runin = false
    },
    today = small,
    figurename = 图,
    contentsname = 目录,
    tablename = 表,
}

\begin{document}

\maketitle

基础软件包括什么?回答这个问题,只需看计算机科学与技术专业的本科生学的四大基础课程:《计算机组成原理》、《数据库原理》、《操作系统》、《编译原理》,也就是说,EDA软件、数据库、操作系统以及编译器是四种最基础的基础软件。

国际上,知名的编译器有 GCC、LLVM等支持众多语言的编译器组合,也有ICC、MSVC 等 C / C++编译器,也有Go、RustC、TSC等各个语言的编译器。现有的『国产』编译器中,比较知名的有提供更为容易上手的GPU编程能力的『太极(Taichi)』\cite{Hu_2020},也有『木兰』等面向嵌入式平台但是由于宣发问题广受批评的编译器。由此可见,对于基础语言(如C++、C)的编译器,我国还是显著依赖于开源编译器,并没有实现真正的『自主』。

那么,使用这些开源编译器『安全』吗?答案当然是不安全。如果不经代码审计而直接使用,无论是编译产物中注入还是在标准库中注入,攻击者都可以轻易构造攻击向量。如,密码学算法大多依赖安全且具有足够熵的随机数,OpenSSL等库均使用自己编写的随机数生成算法而不是语言标准库。

既然如此,那么,是否有必要投入编译器等基础软件的开发呢?我的答案是『有』。对于已有成熟开源产品的应用场景,如编译器或操作系统,我们可以采取审计源码的方式,维护一个可控的分支。对于没有成熟开源产品的领域,如EDA软件,我们则要投入精力开发。

\bibliography{homework-0-reference.bib}

\end{document}
