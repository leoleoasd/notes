\documentclass{ctexart}
\usepackage[T1]{fontenc}
\usepackage[a4paper,top=1.5cm,bottom=1.5cm,left=2cm,right=2cm,marginparwidth=1.75cm]{geometry}
\usepackage{mathtools}
\usepackage{booktabs}
\usepackage{caption}
\usepackage{outlines}
\usepackage[colorlinks=false, allcolors=blue]{hyperref}
\usepackage{os-common}
\renewcommand{\tableautorefname}{表}
\DeclarePairedDelimiter{\set}{\{}{\}}
\DeclarePairedDelimiter{\paren}{(}{)}

\title{第四次操作系统作业}
\author{卢雨轩 19071125}
% \date{\today}
\ctexset{
    section = {
        titleformat = \raggedright,
        name = {,},
        number = \chinese{section}、
    },
    paragraph = {
        runin = false
    },
    today = small,
    figurename = 图,
    contentsname = 目录,
    tablename = 表,
}

\begin{document}

\maketitle

\section*{基础作业}

\begin{outline}[enumerate]
    \1 考虑下面一个系统在某一个时刻的状态。
    \begin{center}
        \begin{tabular}{c c c c}
            & Allocation & Max & Available \\
            & ABCD & ABCD & ABCD \\
            P0 & 0012 & 0012 & 1520 \\
            P1 & 1000 & 1750 \\
            P2 & 1354 & 2356 \\
            P3 & 0632 & 0652 \\
            P4 & 0014 & 0656 \\
        \end{tabular}
    \end{center}
        使用银行家算法回答下面的问题:
        \2 Need矩阵的内容
            \begin{center}
                \begin{tabular}{c c}
                    & Need \\
                    P0 & 0000 \\
                    P1 & 0750 \\
                    P2 & 1002 \\
                    P3 & 0020 \\
                    P4 & 0642 \\
                \end{tabular}
            \end{center}
        \2 系统是否处于安全状态
            
            是
        
        \2 若从进程P1发来一个请求(0,4,2,0),是否可以立即满足?

            可以


\end{outline}

\section*{补充作业}
判断对错:
\begin{outline}
    \1 如果资源分配图中有环,那么就一定有死锁。 

        错误
    \1 死锁的时候系统一定处于非安全状态。
    
        正确
\end{outline}
\end{document}
