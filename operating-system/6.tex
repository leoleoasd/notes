\documentclass{ctexart}
\usepackage[T1]{fontenc}
\usepackage[a4paper,top=1.5cm,bottom=1.5cm,left=2cm,right=2cm,marginparwidth=1.75cm]{geometry}
\usepackage{mathtools}
\usepackage{tikz}
\usepackage{booktabs}
\usepackage{caption}
\usepackage{outlines}
\usepackage{graphicx}
\usepackage{amsthm}
\usepackage{diagbox}
\usepackage{float}
\usepackage[colorlinks=false, allcolors=blue]{hyperref}
\renewcommand{\tableautorefname}{表}
\DeclarePairedDelimiter{\set}{\{}{\}}
\DeclarePairedDelimiter{\paren}{(}{)}
\graphicspath{ {./images/} }

\title{操作系统第六次作业}
\author{卢雨轩 19071125}
% \date{\today}
\ctexset{
    section = {
        titleformat = \raggedright,
        name = {,},
        number = \chinese{section}、
    },
    paragraph = {
        runin = false
    },
    today = small,
    figurename = 图,
    contentsname = 目录,
    tablename = 表,
}

\begin{document}

\maketitle

\section*{基础作业}
\begin{outline}[enumerate]
\1 假设页表在内存保存的分页系统:
\2 如果一次访问内存用200ns,那么在程序里要取出某个地址中的数据需要多少时间?

    400 ns

\2 如果加入有75\%命中率的TLB ,那么内存有效访问时间是多少?

    $400 ns \times 0.25 + 200ns \times 0.75 = 250ns$

\1 有页面访问次序1,2,3,4,2,1,5,6,2,1,2,3,7,6,3,2,1,2,3,6。当页框数是4时,分别使用LRU,FIFO,最优置换算法,分别产生多少次缺页?假设初始时页框都为空。

    \2 LRU:

\begin{table}[H]
    \centering
    \begin{tabular}{c|c|c|c|c|c|c|c|c|c|c|c|c|c|c|c|c|c|c|c|c|}
        \multicolumn{1}{c}{\diagbox{页框号}{访问次序}} & \multicolumn{1}{c}{1} & \multicolumn{1}{c}{2} & \multicolumn{1}{c}{3} & \multicolumn{1}{c}{4} & \multicolumn{1}{l}{2} & \multicolumn{1}{c}{1} & \multicolumn{1}{c}{5} & \multicolumn{1}{c}{6} & \multicolumn{1}{c}{2} & \multicolumn{1}{c}{1} & \multicolumn{1}{c}{2} & \multicolumn{1}{c}{3} & \multicolumn{1}{c}{7} & \multicolumn{1}{c}{6} & \multicolumn{1}{c}{3} & \multicolumn{1}{c}{2} & \multicolumn{1}{c}{1} & \multicolumn{1}{c}{2} & \multicolumn{1}{c}{3} & \multicolumn{1}{c}{6}  \\ 
        \cline{2-21}
        1                                       & 1                     & 1                     & 1                     & 1                     &                       &                       & 1                     & 1                     &                       &                       &                       & 1                     & 1                     & 6                     &                       &                      & 6                     &                       &                         & \\ 
        \cline{2-21}
        2                                       &                       & 2                     & 2                     & 2                     &                       &                       & 2                     & 2                     &                       &                       &                       & 2                     & 2                     & 2                     &                       &                      & 2                     &                       &                        &\\ 
        \cline{2-21}
        3                                       &                       &                       & 3                     & 3                     &                       &                       & 5                     & 5                     &                       &                       &                       & 3                     & 3                     & 3                     &                       &                      & 3                     &                       &                        &\\ 
        \cline{2-21}
        4                                       &                       &                       &                       & 4                     &                       &                       & 4                     & 6                     &                       &                       &                       & 6                     & 7                     & 7                     &                       &                      & 1                     &                       &                        &\\
        \cline{2-21}
        \end{tabular}
\end{table}
        共10次缺页。
    
    \2 FIFO:
    \begin{table}[H]
        \centering
        \begin{tabular}{c|c|c|c|c|c|c|c|c|c|c|c|c|c|c|c|c|c|c|c|c|}
        \multicolumn{1}{c}{\diagbox{页框号}{访问次序}} & \multicolumn{1}{c}{1} & \multicolumn{1}{c}{2} & \multicolumn{1}{c}{3} & \multicolumn{1}{c}{4} & \multicolumn{1}{l}{2} & \multicolumn{1}{c}{1} & \multicolumn{1}{c}{5} & \multicolumn{1}{c}{6} & \multicolumn{1}{c}{2} & \multicolumn{1}{c}{1} & \multicolumn{1}{c}{2} & \multicolumn{1}{c}{3} & \multicolumn{1}{c}{7} & \multicolumn{1}{c}{6} & \multicolumn{1}{c}{3} & \multicolumn{1}{c}{2} & \multicolumn{1}{c}{1} & \multicolumn{1}{c}{2} & \multicolumn{1}{c}{3}  & \multicolumn{1}{c}{6}  \\ 
        \cline{2-21}
        1                                       & 1                     & 1                     & 1                     & 1                     &                       &                       & 5                     & 5                     & 5                     & 5                     &                       & 3                     & 3                     & 3                     &                       & 3                     & 1                     &                       & 1                    &  \\ 
        \cline{2-21}
        2                                       &                       & 2                     & 2                     & 2                     &                       &                       & 2                     & 6                     & 6                     & 6                     &                       & 6                     & 7                     & 7                     &                       & 7                     & 7                     &                       & 3                     & \\ 
        \cline{2-21}
        3                                       &                       &                       & 3                     & 3                     &                       &                       & 3                     & 3                     & 2                     & 2                     &                       & 2                     & 2                     & 6                     &                       & 6                     & 6                     &                       & 6                     & \\ 
        \cline{2-21}
        4                                       &                       &                       &                       & 4                     &                       &                       & 4                     & 4                     & 4                     & 1                     &                       & 1                     & 1                     & 1                     &                       & 2                     & 2                     &                       & 2                      &\\
        \cline{2-21}
        \end{tabular}
        \end{table}
        共14次缺页。
    \2 最优置换算法:
    \begin{table}[H]
    \centering
    \begin{tabular}{c|c|c|c|c|c|c|c|c|c|c|c|c|c|c|c|c|c|c|c|c|}
    \multicolumn{1}{c}{\diagbox{页框号}{访问次序}} & \multicolumn{1}{c}{1} & \multicolumn{1}{c}{2} & \multicolumn{1}{c}{3} & \multicolumn{1}{c}{4} & \multicolumn{1}{l}{2} & \multicolumn{1}{c}{1} & \multicolumn{1}{c}{5} & \multicolumn{1}{c}{6} & \multicolumn{1}{c}{2} & \multicolumn{1}{c}{1} & \multicolumn{1}{c}{2} & \multicolumn{1}{c}{3} & \multicolumn{1}{c}{7} & \multicolumn{1}{c}{6} & \multicolumn{1}{c}{3} & \multicolumn{1}{c}{2} & \multicolumn{1}{c}{1} & \multicolumn{1}{c}{2} & \multicolumn{1}{c}{3} & \multicolumn{1}{c}{6}  \\ 
    \cline{2-21}
    1                                       & 1                     & 1                     & 1                     & 1                     &                       &                       & 1                     & 1                     &                       &                       &                       & 1                     & 7                     &                       &                       &                       & 1                     &                       &                  &     \\ 
    \cline{2-21}
    2                                       &                       & 2                     & 2                     & 2                     &                       &                       & 2                     & 2                     &                       &                       &                       & 2                     & 2                     &                       &                       &                       & 2                     &                       &                   &    \\ 
    \cline{2-21}
    3                                       &                       &                       & 3                     & 3                     &                       &                       & 3                     & 6                     &                       &                       &                       & 6                     & 6                     &                       &                       &                       & 6                     &                       &                    &   \\ 
    \cline{2-21}
    4                                       &                       &                       &                       & 4                     &                       &                       & 5                     & 5                     &                       &                       &                       & 3                     & 3                     &                       &                       &                       & 3                     &                       &                     &  \\
    \cline{2-21}
    \end{tabular}
    \end{table}
    共9次缺页。
\end{outline}
\section*{补充作业}
\begin{outline}[enumerate]
    \1 在一个虚拟存储管理系统中采用页式方法对内存空间进行管理,它有24位的虚拟地址空间,
    而实际的物理地址空间是16位,页框大小为2k。假设有两个进程A和B。其中A进程的0、2页已
    经调入到内存的2、3号页框;B进程的1、3页已经调入到内存的7、8号页框。请问:A进程的虚
    拟地址12FF可以转换成什么物理地址?B进程的虚拟地址17BA可以转换成什么物理地址?如果不
    能转换,操作系统会执行什么操作?

    \texttt{
        0x12ff = 0b0001,0010,1111,1111。页号为0b10,即2号页。映射到了3号页框,因此物理地址为
        0b1,1010,1111,1111 = 0x1aff
    }

    \texttt{
        0x17ba = 0b0001,0111,1011,1010。页号为0b10,即2号页。没有映射到页框,
        会触发缺页中断,由操作系统把一个空闲的页框映射过去。
    }

\end{outline}

\end{document}
