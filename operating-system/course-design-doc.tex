\documentclass{ctexart}
\usepackage[T1]{fontenc}
\usepackage[a4paper,top=1.5cm,bottom=1.5cm,left=2cm,right=2cm,marginparwidth=1.75cm]{geometry}
\usepackage{mathtools}
\usepackage{tikz}
\usepackage{booktabs}
\usepackage{caption}
\usepackage{outlines}
\usepackage{graphicx}
\usepackage{amsthm}
\usepackage{minted}
\usepackage{tabularx}
\usepackage[colorlinks=false, allcolors=blue]{hyperref}
\renewcommand{\tableautorefname}{表}
\DeclarePairedDelimiter{\set}{\{}{\}}
\DeclarePairedDelimiter{\paren}{(}{)}
\graphicspath{ {./images/} }

\title{系统软件课设Pintos项目上机题目与考核须知}
% \author{卢雨轩}
% \date{\today}
\ctexset{
    section = {
        titleformat = \raggedright,
        name = {,},
        number = \chinese{section}、
    },
    paragraph = {
        runin = false
    },
    today = small,
    figurename = 图,
    contentsname = 目录,
    tablename = 表,
}

\begin{document}

\maketitle

\section{Stride调度介绍}

经过了一学期Pintos的『顶峰体验』,你一定受益良多。在Pintos中,我们阅读了默认实现的FCFS算法,并亲手实现了优先级调度、MLFQ调度。但是,以上两种算法均不能控制进程运行时间与其优先级的比例关系。下面,请你以本次上机测试给定的Pintos版本为基础,实现stride调度算法。


\subsection{算法步骤}
\begin{outline}[enumerate]
    \1 为每一个进程设置一个当前stride,表示该进程已经运行的『长度』。另外,设置其对应的pass值(只与进程的优先级有关系),表示进程在调度后,stride需要进行的累加值。
    \1 \emph{每次需要调度时},从当前 ready 态的进程中选择 stride 最小的进程调度。对于获得调度的进程 P,将对应的 stride 加上其对应的步长 pass。
    \1 一个时间片后,回到上一步骤,重新调度当前 stride 最小的进程。
\end{outline}

可以证明,如果令 $\mathrm{P.{pass} = \frac{BigStride}{P.{priority}}}$,其中 P.priority 表示进程的优先级(大于 1 的整数),而 BigStride 表示一个预先定义的大常数,则该调度方案为每个进程分配的时间将与其优先级成正比。证明过程我们在这里略去,有兴趣的同学可以于本次上机测试结束后在网上查找相关资料。

\subsection{算法细节}
\begin{outline}
    \1 stride 调度要求\emph{进程优先级 $priority \ge 2$},所以设定进程优先级 $p \le 1$ 会导致错误。
    \1 进程初始 stride 设置为 0 即可。
\end{outline}

\subsection{注意事项}
在工程实践中,我们会使用固定大小的数据类型(如\texttt{int32\_t})来存储stride,自然,我们会遇到溢出问题。你的算法应该能够在\emph{上列细节的条件}下正确处理溢出后stride的比较,保证每次能够选出不溢出时stride最小的进程。

\section{考核要求}
\subsection{任务说明}
你需要在\emph{我们给定的Pintos版本}上实现stride调度。需要的常数和变量已经定义(\texttt{BIG\_STRIDE}和\texttt{stride})。你需要复用Pintos本身的、用于实现优先级调度的优先级变量(\texttt{struct thread}中的\texttt{priority}域)并保证\texttt{thread\_set\allowbreak\_priority}和\texttt{thread\_get\_priority}工作正常。

你不应该修改\texttt{threads.h}中\texttt{BIG\_STRIDE}的定义。

提示:使用 \mintinline{bash}{git diff 8850fb~2..8850fb src} 命令可以查看我们对于Pintos基础版做出的修改。
\subsection{测试点说明}
考核共有4个测试,在\texttt{threads}文件夹中运行\texttt{make check}即可运行测试(\emph{Pintos Project1 原有的测试已经被删除})。其中,2个测试点是开放的,同学可以在上机测试中查看结果;另外2个测试点是隐藏的,我们会在评测时补充隐藏测试点的内容,在上机考试中,隐藏测试点的结果固定为『测试不通过』。

\begin{table}[h]
    \caption{测试点内容说明}
    \centering
    \begin{tabularx}{\textwidth}{cXc} 
    \toprule
    测试名称            & 测试内容   & 是否隐藏  \\
    \midrule
    stride-one      & 一个进程的情况是否能正确运行                              & 否     \\
    stride-two      & 两个进程的情况下,能否正常运行,且运行时间是否与优先级成正比              & 否     \\
    stride-multiple & 多个进程的情况下,能否正常运行,且运行时间是否与优先级成正比              & 是     \\
    stride-overflow & 多个进程的情况下,且stride可能溢出时,能否正常运行,且运行时间是否与优先级成正比 & 是     \\
    \bottomrule
    \end{tabularx}
\end{table}

\begin{appendix}
\section{GIT操作简要教程}
克隆代码: \mintinline{bash}{git clone --depth=1 <你的项目地址>}

提交代码:\mintinline{bash}{git add .; git commit -a}

推送代码:\mintinline{bash}{git push}
\end{appendix}

\end{document}
