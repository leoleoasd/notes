\documentclass{ctexart}
\usepackage[T1]{fontenc}
\usepackage{mathtools}
\usepackage{booktabs}
\usepackage{caption}
\usepackage{outlines}
\usepackage{graphicx}
\usepackage{os-common}
\usepackage{multicol}
\usepackage[a4paper,top=2.5cm,bottom=2.5cm,left=2cm,right=2cm]{geometry}
\usepackage[colorlinks=false, allcolors=blue]{hyperref}
\renewcommand{\tableautorefname}{表}
\DeclarePairedDelimiter{\set}{\{}{\}}
\DeclarePairedDelimiter{\paren}{(}{)}
\graphicspath{ {./images/} }

\title{操作系统第五次作业}
\author{卢雨轩 19071125}
% \date{\today}
\ctexset{
    section = {
        titleformat = \raggedright,
        name = {,},
        number = \chinese{section}、
    },
    paragraph = {
        runin = false
    },
    today = small,
    figurename = 图,
    contentsname = 目录,
    tablename = 表,
}


% \renewcommand{\headrulewidth}{0.4pt}
% \renewcommand{\headwidth}{\textwidth}
% \renewcommand{\footrulewidth}{0pt}

\begin{document}

\maketitle

\section*{基础作业}

\begin{outline}[enumerate]
    \1 内部碎片与外部碎片之间的区别?
        
        外部碎片指操作系统无法分配给用户进程的内存碎片。

        内部碎片指操作系统分配给用户进程但用户进程没有利用的碎片
    \1 内存按顺序有100k,500k,200k,300k,600k,用首次适应、最佳适应和最差适应如何放置212k,417k,112k,426k的进程?

        \begin{multicols}{3}
            \begin{center}
                \captionof{table}{首次适应}
                \begin{tabular}{c c}
                    \toprule
                    内存需求 & 放置的内存块 \\
                    \midrule
                    212k & 500k \\
                    417k & 600k \\
                    112k & 200k \\
                    426k & --- \\
                    \bottomrule
                \end{tabular}
            \end{center}

            \begin{center}
                \captionof{table}{最佳适应}
                \begin{tabular}{c c}
                    \toprule
                    内存需求 & 放置的内存块 \\
                    \midrule
                    212k & 300k \\
                    417k & 500k \\
                    112k & 200k \\
                    426k & 600k \\
                    \bottomrule
                \end{tabular}
            \end{center}

            \begin{center}
                \captionof{table}{最差适应}
                \begin{tabular}{c c}
                    \toprule
                    内存需求 & 放置的内存块 \\
                    \midrule
                    212k & 600k \\
                    417k & 500k \\
                    112k & 300k \\
                    426k & --- \\
                    \bottomrule
                \end{tabular}
            \end{center}
        \end{multicols}
    \1 假设一个有8个1k页面的逻辑地址空间,映射到一个32个页框的物理内存,问:逻辑地址多少位?物理地址多少位?
        
        逻辑地址:13位 \\
        物理地址:15位
    \1 有段表:
        \begin{center}
            \begin{tabular}{ccc} 
                \toprule
                段 & 基地址  & 长度   \\
                \midrule
                0 & 219  & 600  \\
                1 & 2300 & 14   \\
                2 & 90   & 100  \\
                3 & 1327 & 580  \\
                4 & 1952 & 96   \\
                \bottomrule
                \end{tabular}
        \end{center}
        下面逻辑地址的物理地址是多少?
        \2 0,430; \\
            649
        \2 1,10; \\
            2310
        \2 2,500; \\
            N/A
        \2 3,400; \\
            1727
        \2 4,122 \\
            N/A
    \1 在页面大小为4k的系统中,根据图中所示页表,下面的逻辑地址经过重定位之后的物理地址是什么?
    
    \begin{minipage}{.8\textwidth}
    \begin{enumerate}
        \item 20\\ 
            $addr = 12 \times 4096 + 20 = 49172$
        \item 4100\\
            $addr = 14 \times 4096 + 4 = 57348$
        \item 8300 \\
            $addr = 15 \times 4096 + 108 = 61548$
    \end{enumerate}
    \end{minipage}
    \begin{minipage}{.2\textwidth}
            \begin{tabular}{c|c|} 
                \cline{2-2}
                0 & 12  \\ 
                \cline{2-2}
                1 & 14  \\ 
                \cline{2-2}
                2 & 15  \\ 
                \cline{2-2}
                3 & 10  \\ 
                \cline{2-2}
                4 & 11  \\ 
                \cline{2-2}
                5 & 6   \\ 
                \cline{2-2}
                6 & 13  \\ 
                \cline{2-2}
                7 & 4   \\
                \cline{2-2}
            \end{tabular}
    \end{minipage}
    
    \1 一台计算机为每个进程提供65536字节的地址空间,页面的大小为4k。一个程序有32768字节的正文段,16386字节的数据段,15870字节的堆栈段。问此程序是否能装入此地址空间?若页面大小为512字节呢?
        \2 共有16页。正文段占8页,数据段占5页,堆栈段占4页,无法装入。
        \2 共有128页。正文段占64页,数据段占33页,堆栈段占30页,可以装入。
\end{outline}

\section*{补充作业}
判断对错。
\begin{outline}
    \1 编译时绑定是大多数通用操作系统使用的地址绑定方法。

        正确。
    \1 最佳适配法可以在内存分配过程中留下最小的洞。

        正确。
    \1 为解决内存分配时导致的外部碎片可以采用紧凑的方法来解决,因此需要在地址绑定的时候采用静态重定位方法。

        错误。
    \1 如果现在基地址寄存器的值是1200,界限寄存器的值是350,那么当前进程产生对绝对地址1551的访问是合法的。

        错误。
    \1 可重入代码不可以被共享。

        错误。

\end{outline}
\end{document}
